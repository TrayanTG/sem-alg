%Ще започнем текущата глава от малко по-назад или по-конкретно: ще разгледаме как компютрите са просто детерминирани автомати и как можем да ги обобщим.
\vspace{-0.5cm}
\section{Компютърът - един голям автомат}
\vspace{-0.2cm}
Въпреки, че компютрите обикновено се отъждествяват с детерминирани машини на Тюринг (полезно и от теоретична и от практична гледни точки), те в действителност са детерминирани крайни автомати (тъй като имаме ограничена памет). Нека разгледаме пример.\newline

\begin{examplecp}
	Дадени са следните програми (без вход от потребителя):
	\begin{pseudocode}
		\SetKwData{da}{a}
		
		$Func7():$
		\Mybegin
		{	
			$\da\leftarrow7$\;
			\While{$\da>1$}
			{
				$\da\leftarrow\da-2$\;
			}
			\KwRet{$\da=0$\;}
		}
	\end{pseudocode}
	\begin{pseudocode}
		\SetKwData{da}{a}
		
		$Func4():$
		\Mybegin
		{	
			$\da\leftarrow4$\;
			\While{$\da>1$}
			{
				$\da\leftarrow\da-2$\;
			}
			\KwRet{$\da=0$\;}
		}
	\end{pseudocode}
	\noindent
	Ясно е, че програмите връщат истина т.с.т.к. $a$ е четно и лъжа иначе. Състоянията на автоматите ще бъдат наредени двойки от ред (на програмата) и стойността на променливите (в случая само $a$):
	\begin{itemize}
		\item $a=7$
		\begin{figure}[H]
			\centering
			\scalebox{0.8}{\begin{tikzpicture}[node distance=2cm]				
					\node (17) [state] {$\langle2,?\rangle$};
					\node (27) [state, right of=17] {$\langle3,7\rangle$};
					\node (37) [state, right of=27] {$\langle4,7\rangle$};
					\node (25) [state, right of=37] {$\langle3,5\rangle$};
					\node (35) [state, right of=25] {$\langle4,5\rangle$};
					\node (23) [state, right of=35] {$\langle3,3\rangle$};
					\node (33) [state, right of=23] {$\langle4,3\rangle$};
					\node (21) [state, right of=33] {$\langle3,1\rangle$};
					\node (41) [state, right of=21] {$\langle5,1\rangle$};
					\node (N) [state, right of=41, minimum size=1.4cm] {N};
					
					\draw[arrow] (-1.2,0) -- (17);\draw[arrow] (17) -- (27);
					\draw[arrow] (27) -- (37);\draw[arrow] (37) -- (25);
					\draw[arrow] (25) -- (35);\draw[arrow] (35) -- (23);
					\draw[arrow] (23) -- (33);\draw[arrow] (33) -- (21);
					\draw[arrow] (21) -- (41);\draw[arrow] (41) -- (N);
			\end{tikzpicture}}
		\end{figure}
		
		\item $a=4$
		\begin{figure}[H]
			\centering
			\scalebox{0.8}{\begin{tikzpicture}[node distance=2cm]				
					\node (14) [state] {$\langle2,?\rangle$};
					\node (24) [state, right of=14] {$\langle3,4\rangle$};
					\node (34) [state, right of=24] {$\langle4,4\rangle$};
					\node (22) [state, right of=34] {$\langle3,2\rangle$};
					\node (32) [state, right of=22] {$\langle4,2\rangle$};
					\node (20) [state, right of=32] {$\langle3,0\rangle$};
					\node (40) [state, right of=20] {$\langle5,0\rangle$};
					
					
					\node (Y) [state-final, right of=40, minimum size=1.4cm] {Y};
					
					\draw[arrow] (-1.2,0) -- (14);\draw[arrow] (14) -- (24);
					\draw[arrow] (24) -- (34);\draw[arrow] (34) -- (22);
					\draw[arrow] (22) -- (32);\draw[arrow] (32) -- (20);
					\draw[arrow] (20) -- (40);\draw[arrow] (40) -- (Y);
					
			\end{tikzpicture}}
		\end{figure}
	\end{itemize}
	\noindent
	Макар по дефиниция, ДКА нямат $\varepsilon$-преходи, може да разглеждаме преходите като $\varepsilon$, тъй като от всяко състояние има най-много едно изходно ребро. Разбира се, изложеният метод работи само за предварително фиксирани променливи, т.е. без вход от потребителя. Помислете как бихме могли да обобщим да работи и с вход от потребитля.
\end{examplecp}%\newpage

\section{Недетерминирани алгоритми}

Както вече показахме в предходната секция, всяка програма \textbf{за компютър} може да се представи чрез краен детерминиран автомат. Нека сега да си мислим, че имаме магически компютър, който може да прави много изпълнения едновременно. Лесно се вижда, че такъв магически компютър би се представял чрез краен недетерминиран автомат. Програмите за стандартните компютри описвахме чрез псевдокод. Пргорамите за магическите компютри ще описаваме отново чрез псевдокод, но ще имаме бонус операция - недетерминистичен избор. Разбираме се недетерминираните програми/алгоритми да връщат стойности единствено от $\{$TRUE, FALSE$\}$. Аналогично на ДКА, където дума се разпознава, ако има поне едно успешно изпълнение, тук ще връщаме TRUE, ако има поне едно успешно изпълнение.

\begin{boxdefinition}{Изход на недетерминиран алгоритъм}{}
	Ще казваме, че недетерминиран алгоритъм връща TRUE, т.с.т.к. има поне едно успешно изпълнение.
\end{boxdefinition}

\vspace{0.3cm}\noindent
Нека да разледаме слената задача:
\begin{boxzzr}{COMPOSITES}{composites}
	\dproblem{n\in\mathbb{N}^+}{n\text{ съставно ли е?}}
\end{boxzzr}
\begin{solution}
	Ще покажем детерминирано и недетерминирано решение. Да започнем от детерминираното:
	\begin{pseudocode}
		\SetKwData{dn}{n}
		\SetKwData{dq}{q}
		
		$isCompositeDet(\dn)://\,\dn\in\mathbb{N}^+$
		\Mybegin
		{	
			\Myfor{$\dq\leftarrow2$ \KwTo $\lfloor\sqrt{n}\rfloor$}
			{
				\If{$\dn\equiv0\ (\text{mod}\ q)$}{\KwRet{\True\;}}
			}
			
			\KwRet{\False\;}
		}
	\end{pseudocode}
	Коректност няма да разглеждаме. Ясно как става - чрез инвариант. Времевата сложност на тази програма е $O(\sqrt n)$. Големината на входа е $m=\log_2{n}$, т.е. времевата сложност спрямо големината на входа е $O(\sqrt{2^m})=O\big((\sqrt2)^m\big)$ - експоненциална. А дали има полиномиален (спрямо големината на входа) детерминиран алгоритъм?
	
	\vspace{0.3cm}\noindent
	Нека сега разгледаме недетерминиран алгоритъм, решаващ задачата за полиномиално време:
	\begin{pseudocode}
		\SetKwData{dn}{n}
		\SetKwData{dq}{q}
		
		$isCompositeNonDet(\dn)://\,\dn\in\mathbb{N}^+$
		\Mybegin
		{	
			$\dq\leftarrow\Choose(\{2,3,\dots,\lfloor\sqrt n\rfloor\})$\tcp*{недетерминиран избор}
			\If{$\dn\equiv0\ (\text{mod}\ q)$}{\KwRet{\Success\;}}
			
			\KwRet{\Fail\;}
		}
	\end{pseudocode}
	\begin{remark*}
		Можем да правим недетерминиран избор до предварително избрана константа за константно време. В случая, изборът \textbf{НЕ} е ограничен. Това което се случва отдолу е, че всяко битче на числото се избира да е нула или единица.. тоест недетерминираният избор отнема $\Theta(\log_2{\sqrt n})$ време.
	\end{remark*}
	За да докажем коректността на недетерминирания алгоритъм трябва да покажем, че:
	\begin{itemize}
		\item при съставно $n$: има поне едно успешно изпълнение
		\item при просто $n$: всички изпълнения са неуспешни
	\end{itemize}
	Нека $n$ е съставно. Тогава $\big(\exists q_0\in\{2,3,\dots,\lfloor\sqrt n\rfloor\}\big)\big(n\equiv 0\ (\text{mod}\ q_0)\big)$. Ясно е, че клона на изпълнение, в който choose е избрал $q_0$, е успешен. Нека сега $n$ е просто. Тогава знаем, че $\big(\forall q\in\{2,3,\dots,\lfloor\sqrt n\rfloor\}\big)\big(n\not\equiv 0\ (\text{mod}\ q_0)\big)$. Оттук е ясно, че всички изпълнения са неуспешни. Времевата сложност на тази програма е $\Theta(\log_2{\sqrt n})$. Големината на входа е $m=\log_2{n}$, т.е. времевата сложност спрямо големината на входа е $\Theta(\log_2{\sqrt{2^m}})=\Theta(\frac12\log_2(2^m))=\Theta(\frac m2)=\Theta(m)$ - полиномиална.
\end{solution}

\leavevmode\newline\noindent
Нека сега разгледаме допълнението на горната задача и по-конкретно:
\begin{boxzzr}{PRIMES}{primes}
	\dproblem{n\in\mathbb{N}^+}{n\text{ просто ли е?}}
\end{boxzzr}
\begin{solution}
	На пръв поглед, наивното разсъждение \emph{просто ще върнем отрицанието на резултата от COMPOSITES}, ни върши работа. Нека обаче разгледа по-детайлно:
	\begin{pseudocode}
		\SetKwData{dn}{n}
		
		$isPrimeDetNaive(\dn)://\,\dn\in\mathbb{N}^+$
		\Mybegin
		{	
			\KwRet{$\lnot isCompositeDet(\dn)$\;}
		}
	\end{pseudocode}
	Ясно е, че това решение работи. Също така е ясно, че сложността по време е експоненциална.
	
	\vspace{0.3cm}\noindent
	Нека сега разгледаме наивното за недетерминирания алгоритъм:
	\begin{pseudocode}
		\SetKwData{dn}{n}
		
		$isPrimeNonDetNaive(\dn)://\,\dn\in\mathbb{N}^+$
		\Mybegin
		{	
			\KwRet{$\lnot isCompositeNonDet(\dn)$\;}
		}
	\end{pseudocode}
	За да докажем коректността на недетерминирания алгоритъм трябва да покажем, че:
	\begin{itemize}
		\item при просто $n$: има поне едно успешно изпълнение
		\item при съставно $n$: всички изпълнения са неуспешни
	\end{itemize}
	Ще покажем, че второто изискване не е изпълнено. Наистина, нека $n=16$. Клона на изпълнение, в който $choose(\{2,3,4\})$ ни е избрало 3 е успешен, а ние искахме да е неуспешен. Тоест алгоритъмът не е коректен. Не е тривиално да се измисли полиномиален недетерминиран алгоритъм за тази задача.
\end{solution}
\vspace{0.3cm}
\begin{boxremark*}{}
	Нека $\mathscr{A}$ е задача за разпознаване. Тогава
	\begin{itemize}
		\item Ако имаме детерминиран алгоритъм за $\mathscr{A}$, то тривиално получаваме детерминиран алгоритъм за $\overline{\mathscr{A}}$ със същата времева сложност
		\item Ако имаме недетерминиран алгоритъм за $\mathscr{A}$, то нямаме обща схема по която да получим недетерминиран алгоритъм за $\overline{\mathscr{A}}$
	\end{itemize}
\end{boxremark*}