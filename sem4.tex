\section{Уводни алгоритми}

Ще започнем с това да изтъкнем разликата между изчислителна задача и алгоритъм. Грубо казано, изчислителна задача е нещо общо, а алгоритъмът е прозрачна кутия (т.е. знаем ѝ механизъма на работа), която \emph{"решава"} дадена изчислителна задача. Може да имаме много алгоритми, решаващи една и съща изчислителна задача. За да правим разлика, когато говорим за изчислителна задача ще използваме термините \emph{екземпляр (instance)} и \emph{решение (solution)} вместо \emph{вход} и \emph{изход}. Вторите ще използваме, когато говорим за алгоритъм.\\

\begin{problem}
	Даден е масив $A[1..n]\in(\mathbb{N}_0)^n$ и много на брой заявки от вида:
	\cproblem{i,j\in\{1,\dots,n\}}{\sum\limits_{k=1}^jA[k]}
	Предложете двойка алгоритми (индекс и заявка), решаващи изчислителната задача. Индексирането бива извиквано точно веднъж. Заявки биват извиквани многократно. Заявка не може бъде извикана преди (единственото) извикване на индекса. Разгледайте следния пример, имащ две заявки:
	\cproblemquery{A[1..7]=[4,2,5,7,1,1,10]}{\query{i=2,j=5}{15\ //\text{since }2+5+7+1=15}\\\query{i=5,j=2}{0\ //\text{since the neutral element of }+\text{ is }0}}
\end{problem}

\begin{solution}
	Нека да разгледаме първо индекирането:
	\begin{pseudocode}
		\SetKwData{dA}{A}
		\SetKwData{dn}{n}
		\SetKwData{dSum}{Sum}
		\SetKwData{di}{i}
		
		$Index(\dA[1..\dn])://\,\dA\in(\mathbb{N}_0)^{\dn},\dn\in\mathbb{N}^+$
		\Mybegin
		{
			$\dSum[1..\dn]\leftarrow[0,\dots,0]$\;
			$\dSum[1]\leftarrow\dA[1]$\;
			\Myfor{$i\leftarrow1$ $\KwTo$ $\dn$}
			{
				$\dSum[\di]\leftarrow\dSum[\di-1]+\dA[\di]$\;
			}
			\KwRet{$\dSum[1..\dn]$}\;
		}
	\end{pseudocode}
	След като направихме индекс (връщаният масив $Sum[1..n]$), трябва да го използваме по подходящ начин в заявката:
	\begin{pseudocode}
		\SetKwData{dn}{n}
		\SetKwData{dSum}{Sum}
		\SetKwData{di}{i}
		\SetKwData{dj}{j}
		
		$Query(\dSum[1..\dn],\di,\dj)://\,\dSum\in(\mathbb{N}_0)^{\dn},\di,\dj\in\{1,\dots,\dn\},\dn\in\mathbb{N}^+$
		\Mybegin
		{
			$\dSum[1..\dn]\leftarrow[0,\dots,0]$\;
			
			\If{$\di<1$ \Or $\dj>\dn$}{\KwRet{$-1$};}
			\If{$\di>\dj$}{\KwRet{$0$};}
			\If{$\di=1$}{\KwRet{$\dSum[\dj]$};}
			
			\KwRet{$\dSum[\dj]-\dSum[\di-1]$}\;
		}
	\end{pseudocode}
	Алгоритмите са със сложност по време съответно $\langle\underbrace{\ \ \ \Theta(n)\ \ \ }_{Index(A[1..n])},\underbrace{\quad\quad\Theta(1)\quad\quad}_{Query(Sum[1..n],i,j)}\rangle$.
\end{solution}\vspace{0.5cm}

\begin{problem}\label{prob-4-1}
	Даден е непразен сортиран масив $A[1..n]\in(\{0,1,\dots,n\})^n$ съставен от уникални елементи. Да се намери $k\in\{0,1,\dots,n\}$ такова, че $k$ не е елемент на $A[1..n]$. Предложете колкото можете по-бърз алгоритъм, решаващ изчислителната задача. Разгледайте следния пример:
	\cproblem{A[1..7]=[0,1,2,4,5,6,7]}{3\ //\text{since }3\text{ is not an element of }A[1..7]}
\end{problem}
\begin{solution}
	За разлика от предходната задача, тук нямаме много заявки, а точно една. С други думи може да обединим индекса и заявката в едно общо решение:
	\begin{pseudocode}
		\SetKwData{dA}{A}
		\SetKwData{dn}{n}
		\SetKwData{dleft}{left}
		\SetKwData{dright}{right}
		\SetKwData{dmid}{mid}
		
		$Task2(\dA[1..\dn])://\,\dA\in(\mathbb{N}_0)^n,\dn\in\mathbb{N}^+$
		\Mybegin
		{
			$\dleft\leftarrow1$\;
			$\dright\leftarrow\dn$\;
			
			\While{$\dleft<\dright$}
			{
				$\dmid\leftarrow\lfloor\frac{\dleft+\dright}2\rfloor$\;
				\If{$\dA[\dmid]+1=\dmid$}{$\dleft\leftarrow\dmid+1$\;}
				\Else{$\dright\leftarrow\dmid$\;}
			}
			\If{$\dA[\dn]+1=\dn$}{\KwRet{$\dleft$\;}}
			\KwRet{$\dleft-1$\;}
		}
	\end{pseudocode}
	Алгоритъмът е със сложност по време $\Theta(\log n)$.
\end{solution}\vspace{0.5cm}

\begin{problem}
	Даден е непразен масив $A[1..n]\in\mathbb{Z}^n$ съставен от уникални елементи. Да се намери множеството от наредени тройки $\{\langle i,j,k\rangle\!\in\!\{1,\dots,n\}^3\,|\,i\!<\!j\!<\!k\land A[i]\!+\!A[j]\!+\!A[k]\!=\!0\}$. Предложете колкото можете по-бърз алгоритъм, решаващ изчислителната задача. Разгледайте следния пример:
	\cproblem{A[1..7]=[-10,1,2,3,7,8,9]}{\langle1,2,7\rangle,\langle1,3,6\rangle,\langle1,4,5\rangle}
\end{problem}\newpage
\begin{solution}
	Отново, аналогично на $\probref{prob-4-1}$ нямаме нужда от отделен алгоритъм за индексация и заявка, тъй като имаме само една заявка:
	\begin{pseudocode}
		\SetKwData{dA}{A}
		\SetKwData{dn}{n}
		\SetKwData{dAns}{Ans}
		\SetKwData{di}{i}
		\SetKwData{dj}{j}
		\SetKwData{dk}{k}
		\SetKwData{dorder}{order}
		
		$Task3(\dA[1..\dn])://\,\dA\in\mathbb{Z}^{\dn},\dn\in\mathbb{N}^+$
		\Mybegin
		{
			$sort(\dA[1..\dn],\dorder=ascending)$\;
			$\dAns\leftarrow List.Init()$\tcp*{лист от наредени тройки}

			\Myfor{$\di\leftarrow1$ $\KwTo$ $\dn-2$}
			{
				\If{$\dA[\di]\ge0$}{\Break\;}
				$\dj\leftarrow\di+1$\;
				$\dk\leftarrow\dn$\;
				\While{$\dj<\dk$}
				{
					\If{$\dA[\di]+\dA[\dj]+\dA[\dk]=0$}
					{
						$\dAns.PushBack(\langle\di,\dj,\dk\rangle)$\;
						$\dj\leftarrow\dj+1$\;
						$\dk\leftarrow\dk-1$\;
					}
					\ElseIf{$\dA[\di]+\dA[\dj]+\dA[\dk]<0$}{$\dj\leftarrow\dj+1$\;}
					\Else{$\dk\leftarrow\dk-1$\;}
				}
			}
			\KwRet{$\dAns$}\;
		}
	\end{pseudocode}
	Алгоритъмът е със сложност по време $\Theta(n^2)$.
\end{solution}\leavevmode\newline

\begin{problem}
	Даден е непразен масив $A[1..n]\in\mathbb{R}^n$. Да се сортира възходящо масивът, използвайки единствено функцията $reverse\!:\!\mathbb{R}^n\!\times\!\{1,\dots,n\}\!\times\!\{1,\dots,n\}\!\to\!\mathbb{R}^n$, дефинирана по следния начин:
	\begin{equation*}
		reverse(A[1..n],i,j)\leftrightharpoons\begin{cases}
			\big[A[1],\dots,A[i-1],\underbrace{A[j],A[j-1],\dots,A[i+1],A[i]}_{\text{в обратен ред}},A[j+1],\dots,A[n]\big],\, i<j\\
			A[1..n],\, \text{иначе}\\
		\end{cases}
	\end{equation*}
\end{problem}
\begin{solution}
	Идеята е да вземем най-голям елемент и да го сложим на първа позиция, веднага след което на последната свободна такава (с други думи - сортираме отзад напред). Нека да разиграем малък пример. Нека $A[1..3]=[1,3,2]$. Първо поставяме $3$ на пъра позиция чрез $reverse([1,3,2],1,2)$, т.е. получаваме $[3,1,2]$. След това веднага го слагаме най-отзад чрез $reverse([3,1,2],1,3)$, т.е. получаваме $[2,1,3]$. Така вече най-големият (най-голям в общия случай) елемент си е на позицията. Сега намираме втория най-голям елемент и го слагаме на първа позиция, след което го слагаме на предпоследна позиция чрез последнователно извикване на $reverse([2,1,3],1,1)\mapsto[2,1,3]$ и $reverse([2,1,3],1,2)\mapsto[1,2,3]$. Сега откриваме третият най-голям елемент и го слагаме на първа позиция след което на трета позиция отзад напред (в случая е безсмислено разбира, т.е. последователно извикваме $reverse([1,2,3],1,1)\mapsto[1,2,3]$ и $reverse([1,2,3],1,1)\mapsto[1,2,3]$. Така масивът е сортиран. Както сме свикнали от C/C++ ще си дефинираме и използваме спомагателни функции - за чистота на кода и за по-лесно доказателство на коректността. Нека първо си дефинираме функция, намираща индекса на най-левия максимален елемент от префикс на масива:
	\begin{pseudocode}
		\SetKwData{dA}{A}
		\SetKwData{dn}{n}
		\SetKwData{dmaxIdx}{maxIdx}
		\SetKwData{di}{i}
		\SetKwData{dk}{k}
		
		$findIdxOfMax(\dA[1..\dn],\dk)://\,\dA\in\mathbb{R}^{\dn},\dk\in\{1,\dots,\dn\},\dn\in\mathbb{N}^+$
		\Mybegin
		{
			$\dmaxIdx\leftarrow1$\;
			
			\Myfor{$\di\leftarrow2$ $\KwTo$ $\dk$}
			{
				\If{$\dA[\di]>\dA[\dmaxIdx]$}{$\dmaxIdx\leftarrow\di$\;}
			}
			\KwRet{$\dmaxIdx$\;}
		}
	\end{pseudocode}
	Използвайки тази функция, вече можем чисто да напишем на код идеята от по-горе:
	\begin{pseudocode}
		\SetKwData{dA}{A}
		\SetKwData{dn}{n}
		\SetKwData{dmaxIdx}{maxIdx}
		\SetKwData{dk}{k}
		
		$Task4(\dA[1..\dn])://\,\dA\in\mathbb{R}^{\dn},\dn\in\mathbb{N}^+$
		\Mybegin
		{	
			\Myfor{$\dk\leftarrow\dn$ $\KwDownTo$ $2$}
			{
				$\dmaxIdx\leftarrow findIdxOfMax(\dA[1..\dn],\dk)$\;
				$reverse(\dA[1..\dn],1,\dmaxIdx)$\;
				$reverse(\dA[1..\dn],1,\dk)$\;
			}
			\KwRet{$\dA[1..\dn]$\;}
		}
	\end{pseudocode}
	Алгоритъмът е със сложност по време $\Theta(n^2)$.
\end{solution}\leavevmode\newline

\begin{problem}
	Нека е дадена непразна булева матрица $A[1..m][1..n]\in(\{0,1\}^n)^m$, чиито колонки са сортирани низходящо, т.е. $\big(\forall r\in\{1,\dots,m-1\}\big)\big(\forall c\in\{1,\dots,n\}\big)\big(A[r+1][c]\le A[r][c]\big)$. Да се намери височината на най-високата колонка от единици, т.е. $\max\{\sum_{r=1}^m\!A[r][c]\,|\,c\in\{1,\dots,n\}\}$. Предложете колкото можете по-бърз алгоритъм, решаващ изчислителната задача. Разгледайте следния пример:
	\cproblem{A[1..4][1..5]=\begin{bmatrix}
							1 & 0 & 1 & 1 & 0\\
							0 & 0 & 1 & 1 & 0\\
							0 & 0 & 1 & 0 & 0\\
							0 & 0 & 0 & 0 & 0
							\end{bmatrix}}{3\ //\text{since that is the highest row of ones}}
\end{problem}
\begin{solution}
	Идеята е много проста. Откриваме височината на първата колонка и я запазваме в променлива $h$. След това проверяваме дали втроата колонка би имала повече единици. Ако има, то преглеждаме нейната височина и обновяваме $h$. Наблюдението, което използваме е, че тъй като търсим максимална по височина колонка, то ако втората колонка има по-малка височина от първата, то директно я прескачаме. Така се движим "стъпаловидно"$\ $от горе-ляво към долу-дясно. За нагледно с червен цвят е отбелязан пътят, който ще "измине"$\ $алгоритъмът на примерния вход:
	\begin{center}
		$\begin{bmatrix}
			\textcolor{red}1 & 0 & 1 & 1 & 0\\
			\textcolor{red}0 & \textcolor{red}0 & \textcolor{red}1 & 1 & 0\\
			0 & 0 & \textcolor{red}1 & 0 & 0\\
			0 & 0 & \textcolor{red}0 & \textcolor{red}0 & \textcolor{red}0
		\end{bmatrix}$
	\end{center}
	Ясно е, че клетките над червената начупена линия няма нужда да бъдат разглеждани. Вече, въоражени с идея, сме готови да реализираме алгоритъм, решаващ изчислителната задача:
	\begin{pseudocode}
		\SetKwData{dA}{A}
		\SetKwData{dn}{n}
		\SetKwData{dm}{m}
		\SetKwData{di}{i}
		\SetKwData{dj}{j}
		
		$Task5(\dA[1..\dm][1..\dn])://\,\dA\in(\{0,1\}^{\dn})^{\dm},\dn,\dm\in\mathbb{N}^+$
		\Mybegin
		{	
			$\di\leftarrow0$\;
			\Myfor{$\dj\leftarrow1$ $\KwTo$ $\dn$}
			{
				\While{$\di<\dm$ \And $\dA[\di+1][\dj]=1$}{$\di\leftarrow\di+1$\;}
			}
			\KwRet{$\di$\;}
		}
	\end{pseudocode}
	Алгоритъмът е със сложност по време $\Theta(n)+O(m)=O(n+m)$.
\end{solution}\vspace{0.35cm}

\begin{problem}
	Нека е дадена непразна матрица $A[1..m][1..n]\in((\mathbb{N}_0)^n)^m$ от уникални елементи, със свойството, че всеки елемент е по-голям от северните и западните си съседи. Казано формално $\forall r\forall c\big(A[i][j]<A[i+1][j]\land A[i][j]<A[i][j+1]\big)$, за $r$ и $c$ подходящи: да не излизаме извън дефиниционната област. Нека също така е дадено число $k\in\mathbb{N}_0$. Да се провери дали елемента $k$ се среща в матрицата $A[1..m][1..n]$. Предложете колкото можете по-бърз алгоритъм, решаващ изчислителната задача. Разгледайте следния пример:
	\cproblem{A[1..4][1..5]=\begin{bmatrix}
							 1 &  2 &  7 & 12 & 14\\
							 3 &  6 &  8 & 16 & 17\\
							 5 & 10 & 11 & 20 & 21\\
							13 & 15 & 19 & 22 & 25
	\end{bmatrix},k=15}{\text{TRUE}}
\end{problem}
\begin{solution}
	Идеята е да започнем или в най-североизточният или най-югозападният край и стъпаловидно да търсим елемента. Нека за определеност започваме в североизточният ъгъл. Движим се надолу, когато търсения елемент е по-голям от текущия и наляво, ако търсеният елемент е по-малък. Разбира се, ако текущия и търсения елемент съвпадат връщаме \text{TRUE}. Нека да оцветим в червено пътят, който изминава алгоритъмът за да намери 15 в примерната матрица:
	\begin{center}
		$\begin{bmatrix}
			1 &  2 &  7 & 12 & \textcolor{red}{14}\\
			3 &  6 &  \textcolor{red}8 & \textcolor{red}{16} & \textcolor{red}{17}\\
			5 & 10 & \textcolor{red}{11} & 20 & 21\\
			13 & \textcolor{red}{15} & \textcolor{red}{19} & 22 & 25
		 \end{bmatrix}$
	\end{center}
	Защо тази идея работи се убедете за упражнение! Може да го докажете дори формално. Вече сме готови да реализираме алгоритъм, решаващ изчислителната задача:
	\begin{pseudocode}
		\SetKwData{dA}{A}
		\SetKwData{dm}{m}
		\SetKwData{dn}{n}
		\SetKwData{di}{i}
		\SetKwData{dj}{j}
		\SetKwData{dk}{k}
		
		$Task6(\dA[1..\dm][1..\dn],\dk)://\,\dA\in((\mathbb{N}_0)^n)^m,\dn,\dm\in\mathbb{N}^+$
		\Mybegin
		{
			$\di\leftarrow1$\;
			$\dj\leftarrow\dn$\;
			
			\While{$\di\le\dm$ \And $\dj\ge1$}
			{
				\If{$\dA[\di][\dj]=\dk$}{\KwRet{\True\;}}
				\ElseIf{$\dA[\di][\dj]>\dk$}{$\dj\leftarrow\dj-1$\;}
				\Else{$\di\leftarrow\di+1$\;}
			}
			\KwRet{\False\;}
		}
	\end{pseudocode}
	Алгоритъмът е със сложност по време $O(n+m)$.
\end{solution}\newpage

\subsection{Добавяне на фиктивни променливи}

Доказателството за коректност на алгоритъм (зададен чрез псевдокод) не винаги е възможно. Оказва се, че не ни достигат променливите, дадени ни в псевдокода. В зависимост от това, къде се е провалило доказателството, добавяме нови, $\textbf{фиктивни}$ променливи. Те не оказват никакво влияние на алгоритъма. Служат единствено за опорни точки в доказателството за коректност. Независимо дали алгоритъмът е итеративен или рекурсивен, такъв проблем може да настъпи. Сега ще се съсредоточим върху итеративни алгоритми.\\
\emph{Спорно е дали мястото на подсекцията е тук или като подсекция на $\secref{sec:iter-alg-correctness}$.}\newline

\begin{problem}
	Даден е непразен масив $A[1..n]\in\mathbb{Z}^n$. Да се състави колкото се може по-бърз алгоритъм, намиращ мажорантата на $A[1..n]$, ако такава има. Съставете втори алгоритъм, който да установява дали мажоранта има.
\end{problem}
\begin{solution}
	Преди да изложим какъв да е псевдокод, ще докажем спомагателно твърдение. Въвеждаме следните означения:
	\begin{boxlocaldefinition*}{}{}
		\begin{itemize}
			\item $pair(p)\overset{\text{def}}\leftrightarrow\big(|p|=2\big)$
			\item $pset(S)\overset{\text{def}}\leftrightarrow\big(\forall p\in S\big)\big(pair(p)\big)$
			\item $cpair_{A[1..n]}(i,j)\!\overset{\text{def}}\leftrightarrow\!\big(i\!\in\!\{1,\dots,n\}\big)\!\land\!\big(j\!\in\!\{1,\dots,n\}\big)\!\land\!\big(A[i]\!\ne\! A[j]\big)$ //i.e. contrary pair
			\item $comp_{A[1..n]}(\langle i,j\rangle,\langle i',j'\rangle)\!\overset{\text{def}}\leftrightarrow\!\big(i\ne i'\big)\!\land\!\big(i\ne j'\big)\!\land\!\big(j\ne i'\big)\!\land\!\big(j\ne j'\big)$ //i.e. compatible
			\item $cpset_{A[1..n]}(S)\overset{\text{def}}\leftrightarrow pset(S)\land\big(\forall\langle i,j\rangle\in S\big)\big(cpair_{A[1..n]}(i,j)\big)\land$
			
			\hspace{3.2cm}$\big(\forall p_1\in S\big)\big(\forall p_2\in S\big)\big(comp_{A[1..n]}(p_1,p_2)\big)$
			\item $mcpset_{A[1..n]}(S)\overset{\text{def}}\leftrightarrow cpset_{A[1..n]}(S)\land\forall i\forall j\big(\langle i,j\rangle\notin S\Rightarrow\lnot cpset_{A[1..n]}(S\cup\{\langle i,j\rangle\})\big)$
			\item $idx(S)\leftrightharpoons Dom(S)\cup Rng(S)$ //also known as field
		\end{itemize}
	\end{boxlocaldefinition*}
	\begin{remark*}
		Казано неформално:
		\begin{itemize}
			\item ц-чифтове правим на индекси от $A[1..n]$, съответстващи на различни елементи
			\item ц-чифтове $\langle i,j\rangle$ и $\langle i',j'\rangle$ наричаме съвместими, ако $|\{i,i',j,j'\}|=4$
			\item множество от съвместими ц-чифтове наричаме максимално, ако не можем да добавим ц-чифт, така че да остане съвместимо (т.е. всеки два елемента са съвместими)
		\end{itemize}
	\end{remark*}%\leavevmode\newline
	\vspace{0.3cm}
	\begin{boxproposition}{}{majority-element}
		Нека $A[1..n]$ има мажоранта $m$. Тогава:
		\begin{itemize}
			\item $\forall S\big(mcpset_{A[1..n]}(S)\Rightarrow idx(S)\ne\{1,\dots,n\}\big)$
			\item $\forall S\Big(mcpset_{A[1..n]}(S)\Rightarrow\big(\forall i\!\in\!\{1,\dots,n\}\!\setminus\! idx(S)\big)\big(A[i]=m\big)\Big)$
		\end{itemize}
	\end{boxproposition}
	\begin{remark*}
		Казано неформално:
		\begin{itemize}
			\item за всяко максимално множество от съвместими ц-чифтове, има нечифтован индекс
			\item за всяко максимално множество от съвместими ц-чифтове, всеки нечифтован индекс отговаря на елемент, равен на мажорантата
		\end{itemize}
	\end{remark*}

	\begin{proof}
		Нека $A[1..n]$ е произволен масив, съдържащ мажоранта $m$. Нека $S$ е множество: $mcpset_{A[1..n]}(S)$ и нека ознчим броят на срещанията на мажорантата с $k$. Нека $i_1,\dots,i_k$ са (точно) индексите на срещанията на мажорантата в $A[1..n]$. Допускаме, че индексите на всички срещания на мажорантата са ц-чифтовани, т.е. има $k$ различни (от $i_1,\dots,i_k$ и помежду си) индекси $j_1,\dots,j_k$ такива, че $\{\langle i_1,j_1\rangle,\dots,\langle i_k,j_k\rangle\}\subseteq S$. Тоест $A[1..n]$ се състои от поне $2k$ елементи, или казано с други думи $n\ge2k$. От друга страна, знаем че $k\ge\lfloor\frac n2\rfloor+1>\frac n2$, откъдето $2k>n$, което е абсурд. Следователно има поне един не ц-чифтован индекс $i$, отговарящ на мажорантата. Нека сега допуснем, че има не ц-чифтован индекс $j$, отговарящ на елемент, различен от мажорантата. Тогава $cpset_{A[1..n]}(S\cup\{\langle i,j\rangle\})$, което противоречи с максималността на $S$. 
	\end{proof}

	\vspace{0.3cm}

	\begin{examplecp}
		Нека $A[1..10]=[1,2,3,1,1,1,4,1,1,5]$. Примери за максимални множества от съвместими ц-чифрове са $S_1=\{\langle1,2\rangle,\langle3,4\rangle,\langle5,7\rangle,\langle6,10\rangle\},S_2=\{\langle2,3\rangle,\langle7,10\rangle\}$. В първия случай множеството от не ц-нечифтованите индекси е $\{8,9\}$, а във втория е $\{1,4,5,6,8,9\}$. И за двете множества $S_1$ и $S_2$ имаме, че:
		\begin{itemize}
			\item имат не ц-чифтован индекс, т.е. $idx(S_1)=\{1,2,3,4,5,6,7,10\}\ne\{1,\dots,10\}$ и $idx(S_2)=\{2,3,7,10\}\ne\{1,\dots,10\}$;
			\item всички не ц-чифтовани индекси отговарят на елемент, равен на мажорантата $m=1$, т.е. за $S_1$ имаме $A[8]=A[9]=m$, а за $S_2$ имаме $A[1]=A[4]=A[5]=A[6]=A[8]=A[9]=m$.
		\end{itemize}
	\end{examplecp}\noindent\newline

	\noindent
	Първоначално ще изложим алгоритъм, намиращ мажорантата, при условие, че входният масив съдържа мажоранта:
	\begin{pseudocode}
		\SetKwData{dA}{A}
		\SetKwData{dn}{n}
		\SetKwData{dcurr}{curr}
		\SetKwData{dcnt}{cnt}
		\SetKwData{di}{i}
		
		$MajorityElement(\dA[1..\dn])://\,\dA\in\mathbb{Z}^n,\dn\in\mathbb{N}^+,A\text{ has majority element}$
		\Mybegin
		{	
			$\dcurr\leftarrow1$\;
			$\dcnt\leftarrow1$\;
			\Myfor{$\di\leftarrow2$ $\KwTo$ $\dn$}
			{
				\If{$\dA[\dcurr]=\dA[\di]$}{$\dcnt\leftarrow\dcnt+1$\;}
				\Else
				{
					$\dcnt\leftarrow\dcnt-1$\;
					\If{$\dcnt=0$}
					{
						$\dcurr\leftarrow\di$\;
						$\dcnt\leftarrow1$\;
					}
				}
			}
			\KwRet{$\dA[\dcurr]$\;}
		}
	\end{pseudocode}
	Сега ще се опитаме да докажем коректността на алгоритъма (няма да успеем).
	\begin{boxinvariant*}{}{}
		При всяко $k$-то достигане на ред $4$ имаме, че
		\begin{itemize}
			\item $i=k+1$
			\item $curr\in\{1,\dots,i-1\}$
			\item $cnt>0$
			\vspace{-0.4cm}
			\item $\exists S\big(mcpset_{A[1..i-1]}(S)\land\overbrace{(cnt>0\Rightarrow curr\notin idx(S))}^{\equiv\ curr\,\notin\, idx(S)}\land(\overbrace{\ \ \ cnt\ \ \ }^{\text{не ц-чифт.}}+\overbrace{|idx(S)|}^{\text{ц-чифт.}}=n)\big)$
		\end{itemize}
	\end{boxinvariant*}
	\begin{remark*}
		Казано неформално: има максимално множество от съвместими ц-чифтове, където индексите са на $A[1..i-1]$, в което $curr$ не е ц-чифтован и броят не ц-чифтовани индекси на $A[1..i-1]$ е $cnt>0$.
	\end{remark*}
	\begin{base}
		При $k=1$-во достигане на ред $4$ имаме, че $i=2$, $curr\!=\!1\!\in\!\{1,\dots,2-1\}\!=\!\{1,\dots,i-1\}$. Разглеждаме подмасивът $A[1..1]$. Той е съставен от един елемент. Тогава $S=\emptyset$ ни е свидетел. Наистина, $1\notin\emptyset$ и има $cnt=1>0$ не ц-чифтовани индекси. 
	\end{base}
	\begin{maintenance}
		Нека е вярно за някое $k$-то непоследно достигане на ред 4 и нека $S_{old}$ ни е свидетел, т.е. $mcpset_{A[1..i_{old}-1]}(S_{old})\land(cnt_{old}>0\Rightarrow curr_{old}\notin idx(S_{old}))\land(cnt_{old}+|idx(S_{old})|=n)$. Сега се изпълнява за $k$-ти път тялото на цикъла. Имаме $i_{new}=i_{old}+1=k+2$.
		\begin{mycase}
			\item $(A[curr_{old}]=A[i_{old}])$

			\vspace{-0.35cm}
			Тогава $curr_{new}=curr_{old}\in\{1,\dots,i_{old}-1\}\subset\{1,\dots,\overbrace{i_{new}-1}^{i_{old}}\}$, $cnt_{new}=cnt_{old}+1$. От ИП имаме, че $S_{old}$ е максимално множество от съвместими ц-чифтове, където индексите са на $A[1..i_{old}-1]$, $curr_{old}$, не е ц-чифтован и броят не ц-чифтовани индекси на $A[1..i_{old}-1]$ е $cnt_{old}>0$. Лесно се проверява, че $S_{old}$ е максимално множество от съвместими ц-чифтове, където индексите са на $A[1..i_{new}-1]$, $curr_{new}$ не е ц-чифтован и броят не ц-чифтовани инекси на $A[1..i_{new}-1]$ е $cnt_{new}>1>0$, т.е. $S_{old}$ е свидетел: $mcpset_{A[1..i_{new}-1]}(S_{old})\land(cnt_{new}>0\Rightarrow curr_{new}\notin idx(S_{old}))\land(cnt_{new}+|idx(S_{old})|=n)$.
			
			\item $(A[curr_{old}]\ne A[i_{old}])$
			
			От ИП имаме, че $S_{old}$ е максимално множество от съвместими ц-чифтове, където индексите са на $A[1..i_{old}-1]$, $curr_{old}$ не е ц-чифтован и броят не ц-чифтовани индекси на $A[1..i_{old}-1]$ е $cnt_{old}>0$. Тъй като $S_{old}$ е максимално, то всички не ц-чифтовани индекси отговарят на елементи с една и съща стойност $A[curr_{old}]\ne A[i_{old}]$.
			\begin{mycase}
				\item $(cnt_{old}>1)$
				
				\vspace{-0.35cm}
				Тогава $cnt_{new}=cnt_{old}-1>0$, $curr_{new}=curr_{old}\in\{1,\dots,i_{old}-1\}\subset\{1,\dots,\overbrace{i_{new}-1}^{i_{old}}\}$. Нека $j\ne curr_{old}$ е индекс на $A[1..i_{old}-1]$, който не е ц-чифтован. Такъв има поради горното разъждение и защото $cnt_{old}>1$. Нещо повече, $A[j]\ne A[i_{old}]$. Дефинираме $S_{new}\leftrightharpoons S_{old}\cup\{\langle j,i_{old}\rangle\}$. Лесно се проверява, че $S_{new}$ е максимално множество от съвместими ц-чифтове, където индексите са на $A[1..i_{new}-1]$, $curr_{new}$ не е ц-чифтован и броят не ц-чифтованите индекси на $A[1..i_{new}-1]$ е $cnt_{new}$, т.е. $S_{new}$ е търсеният свидетел.
				
				\item $(cnt_{old}=1)$
				
				Тогава $cnt_{new}=1$, $curr_{new}=i_{old}\in\{1,\dots,i_{old}\}=\{1,\dots,i_{new}-1\}$. Единствените два не ц-чифтовани индекса на $A[1..i_{old}]$ са $curr_{old}$ и $i_{old}$ и имаме, че $A[curr_{old}]\ne A[i_{old}]$. Дефинираме $S_{new}\leftrightharpoons S_{old}\cup\{\langle curr_{old},i_{old}\rangle\}$. Лесно се проверява, че $S_{new}$ е максимално множество от съвместими ц-чифтове, където индексите са на $A[1..i_{new}-1]$, но $cnt_{new}>0$ и $curr_{new}=i_{old}$ е ц-чифтован (а ние искахме да докажем импликацията $cnt_{new}>0\Rightarrow curr_{new}$ е не ц-чифтован). \textbf{Не успяхме да намерим свидетел}. В общия случай, такъв може да има, но ние просто да не сме успели да го намерим (в конкретната задача може да се докаже, че наистина няма).
			\end{mycase}
		\end{mycase}
	Това всъщност беше очакван резултат, тъй като при вход $A[1..4]=[1,2,1,1]$, още след първата итерация на цикъла ц-чифтоваме индексите $1$ и $2$ и оставаме без не ц-чифтовани индекси, където индексите са на масива $A[1..2]$.
	\end{maintenance}

	\noindent
	Нека сега разгледаме следната преработка на превдокода:
	\begin{pseudocode}
		\SetKwData{dA}{A}
		\SetKwData{dn}{n}
		\SetKwData{dcurr}{curr}
		\SetKwData{dcnt}{cnt}
		\SetKwData{di}{i}
		\SetKwData{dt}{t}
				
		$MajorityElementExpanded(\dA[1..\dn])://\,\dA\in\mathbb{Z}^n,\dn\in\mathbb{N}^+,A\text{ has majority element}$
		\Mybegin
		{	
			$\dcurr\leftarrow1$\;
			$\dcnt\leftarrow1$\;
			\textcolor{mygreen}{$\dt\leftarrow0$\;}
			\Myfor{$\di\leftarrow2$ $\KwTo$ $\dn$}
			{
				\If{$\dA[\dcurr]=\dA[\di]$}{$\dcnt\leftarrow\dcnt+1$\;}
				\Else
				{
					$\dcnt\leftarrow\dcnt-1$\;
					\If{$\dcnt=0$}
					{
						$\dcurr\leftarrow\di$\;
						$\dcnt\leftarrow1$\;
						\textcolor{mygreen}{$\dt\leftarrow1-\dt$\;}
					}
				}
			}
			\KwRet{$\dA[\dcurr]$\;}
		}
	\end{pseudocode}
	Ясно е, че променливата $t$ не оказва влияние на останалите променливи и на изпълнението на кода изобщо. С други думи, ясно е, че $t$ е фиктивна променлива. В някои ситуации, където не е толкова тривиално, е необходимо да се направи доказателство с инвариант, за еквивалентността на променливите за всяко $k$-то достигане на ред 4 и ред 5 съответно за първоначалния и модифицирания алгоритъм. Вече сме готови да формулираме инвариант.
		\begin{boxinvariant*}{}{}
		При всяко $k$-то достигане на ред $5$ имаме, че
		\begin{itemize}
			\item $i=k+1$
			\item $curr\in\{1,\dots,i-1\}$
			\item $cnt>0$
			\item $t\in\{0,1\}$
			\vspace{-0.5cm}
			\item $\exists S\big(mcpset_{A[1..i-1]}(S)\!\land\!\overbrace{(cnt-t>0\!\Rightarrow\! curr\!\notin\! idx(S))}^{\text{ако има не ц-чифт., то }curr\text{ е такъв}}\!\land(\overbrace{cnt-t}^{\text{не ц-чифт.}}\!+\!\overbrace{|idx(S)|}^{\text{ц-чифт.}}\!=\!n)\big)$
		\end{itemize}
	\end{boxinvariant*}
	\begin{remark*}
		За разлика от първоначалния инватирант, тук антецедента $cnt-t>0$ на импликацията $cnt-t>0\Rightarrow curr\notin idx(S)$ не е винаги $true$.
	\end{remark*}
	\begin{base}
		При $k\!=\!1$-во достигане на ред $5$ имаме, че $i\!=\!2$, $curr\!=\!1\!\in\!\{1\}$, $cnt\!=\!1\!>\!0$, $t\!=\!0\!\in\!\{0,1\}$. Разглеждаме подмасивът $A[1..1]$. Той е съставен от един елемент. Тогава $S=\emptyset$ ни е свидетел. Наистина, $1\!\notin\!\emptyset$ и има $cnt\!-\!t\!=\!1\!-\!0\!=\!1\!>\!0$ не ц-чифтовани индекси. 
	\end{base}
	\begin{maintenance}
		Нека е вярно за някое $k$-то непоследно достигане на ред $5$ и нека $S_{old}$ е такова, че $mcpset_{A[1..i_{old}-1]}(S_{old})\land(cnt_{old}-t_{old}>0\Rightarrow curr_{old}\notin idx(S_{old}))\land(cnt_{old}-t_{old}+|idx(S_{old})|=n)$. Сега се изпълнява за $k$-ти път тялото на цикъла. Имаме $i_{new}=i_{old}\!+\!1=k\!+\!2$. Колкото до $t$, имаме $t_{new}=t_{old}$ или $t_{new}=1-t_{old}$, т.е. $t_{new}\in\{0,1\}$.
		\begin{mycase}
			\item $(A[curr_{old}]=A[i_{old}])$
			
			\vspace{-0.35cm}
			Тогава $curr_{new}=curr_{old}\in\{1,\dots,i_{old}-1\}\subset\{1,\dots,\overbrace{i_{new}-1}^{i_{old}}\}$, $cnt_{new}=cnt_{old}+1$ и $t_{new}=t_{old}$. От ИП имаме, че $S_{old}$ е максимално множество от съвместими ц-чифтове, където индексите са на $A[1..i_{old}-1]$, $curr_{old}$ не е ц-чифтован (при условие, че има не ц-чифтовани индекси) и броят не ц-чифтовани индекси на $A[1..i_{old}-1]$ е $cnt_{old}-t_{old}$.
			\begin{mycase}
				\item $(cnt_{old}-t_{old}=0)$
				
				Тогава $curr_{old}$ е чифтован спрямо $S_{old}$. Нека $p\in S_{old}$ е ц-чифта, който съдържа $curr_{old}$. Без ограничение на общността, нека $curr_{old}$ се среща в дясната координата на $p$, тоест $p=\langle j,curr_{old}\rangle$, където $j\in\{1,\dots,i_{old}-1\}:\ A[j]\ne A[curr_{old}]$. Дефинираме $S_{new}\leftrightharpoons (S_{old}\setminus\{p\})\cup\{\langle j,i_{old}\rangle\}$. Лесно се проверява, че $S_{new}$ е максимално множество от съвместими ц-чифтове, където индексите са на $A[1..i_{new}-1]$, $curr_{new}=curr_{old}$ не е ц-чифтован спрямо $S_{new}$ (тук условието да имаме поне един нечифтован елемент е изпълнено) и броят не ц-чифтовани елементи на $A[1..i_{new}-1]$ е $cnt_{new}-t_{new}=1$. Тоест, $S_{new}$ е свидетел: $mcpset_{A[1..i_{new}-1]}(S_{new})\land(cnt_{new}>t_{new}\Rightarrow curr_{new}\notin idx(S_{new}))\land(cnt_{new}-t_{new}+|idx(S_{new})|=n)$.
				
				\item $(cnt_{old}-t_{old}>0)$
				
				Тогава $curr_{new}=curr_{old}$ не е ц-чифтован (спрямо $S_{old}$). Сега, лесно се проверява, че $S_{old}$ е максимално множество от съвместими ц-чифтове, където индексите са на $A[1..i_{new}-1]$, $curr_{new}$ не е ц-чифтован (тук условието да имаме поне един нечифтован елемент е изпълнено) и броят не ц-чифтовани елементи на $A[1..i_{new}-1]$ е $cnt_{new}-t_{new}$. С други думи, $S_{old}$ е свидетел: $mcpset_{A[1..i_{new}-1]}(S_{old})\land(cnt_{new}>t_{new}\Rightarrow curr_{new}\notin idx(S_{old}))\land(cnt_{new}-t_{new}+|idx(S_{old})|=n)$.
				
			\end{mycase}
		
			\item $(A[curr_{old}]\ne A[i_{old}])$
			
			От ИП имаме, че $S_{old}$ е максимално множество от съвместими ц-чифтове, където индексите са на $A[1..i_{old}-1]$, $curr_{old}$ не е ц-чифтован спрямо $S_{old}$ (при условие, че има не ц-чифтовани индекси) и броят не ц-чифтовани индекси на $A[1..i_{old}-1]$ е $cnt_{old}-t_{old}$. Тъй като $S_{old}$ е максимално, то всички не ц-чифтовани индекси отговарят на елементи с една и съща стойност $A[curr_{old}]\ne A[i_{old}]$.
			\begin{mycase}
				\item $(cnt_{old}-t_{old}>1)$ //$cnt_{old}>t_{old}+1\ge1$, т.е. $cnt_{old}>1$
				
				\vspace{-0.35cm}
				Тогава $cnt_{new}=cnt_{old}-1>0$, $curr_{new}=curr_{old}\in\{1,\dots,i_{old}-1\}\subset\{1,\dots,\overbrace{i_{new}-1}^{i_{old}}\}$ и $t_{new}=t_{old}$. Нека $j\ne curr_{old}+t_{old}$ е индекс, който не е ц-чифтован. Такъв има поради горното разъждение и защото $cnt_{old}-t_{old}>1$. Нещо повече, $A[j]\ne A[i_{old}]$. Дефинираме $S_{new}\leftrightharpoons S_{old}\cup\{\langle j,i_{old}\rangle\}$. Лесно се проверява, че $S_{new}$ е максимално множество от съвместими ц-чифтове, където индексите са на $A[1..i_{new}-1]$, $curr_{new}$ не е ц-чифтован спрямо $S_{new}$ (тук условието да имаме поне един нечифтован елемент е изпълнено) и броят на ц-чифтованите елементи на $A[1..i_{new}-1]$ е $cnt_{new}-t_{new}=cnt_{old}-t_{old}-1$, т.е. $S_{new}$ е търсеният свидетел.
				
				\item $(cnt_{old}-t_{old}=1)$ //$cnt_{old}=t_{old}+1\le2$, но $cnt_{old}>0$ т.е. $cnt_{old}\in\{1,2\}$
				
				Тогава $cnt_{new}=t_{new}=1$ и $curr_{new}\in\{1,\dots,i_{new}-1\}$. Наистина, от ИП имаме, че $i_{old}\in\{1,2\}$ и $t_{old}\in\{0,1\}$ и директно проверяваме следните два подслучая:
				\begin{mycase}
					\item $(cnt_{old}=1\land t_{old}=0)$
					\item $(cnt_{old}=2\land t_{old}=1)$
				\end{mycase}
				Дефинираме $S_{new}\leftrightharpoons S_{old}\cup\{\langle curr_{old},i_{old}\rangle\}$. Лесно се проверява, че $S_{new}$ е максимално множество от съвместими ц-чифтове, където индексите са от $A[1..i_{new}-1]$ и има $cnt_{new}-t_{new}=0$ не ц-чифтовани индекси спрямо $S_{new}$ от $A[1..i_{new}-1]$.
				
				\item $(cnt_{old}-t_{old}=0)$ //$cnt_{old}=t_{old}=1$
				
				Тогава $cnt_{new}=1$, $curr_{new}=i_{old}\in\{1,\dots,i_{old}\}=\{1,\dots,i_{new}-1\}$ и $t_{new}=1-t_{old}=0$. Лесно се проверява, че $S_{old}$ е максимално множество от съвместими ц-чифтове, където индексите са от $A[1..i_{new}-1]$ и има $cnt_{new}-t_{new}=1$ не ц-чифтован индекс на $A[1..i_{new}-1]$ и при това той е $curr_{new}$.
			\end{mycase}
		\end{mycase}			
	\end{maintenance}
	\begin{termination}
		При $k=n$-то достигане на ред 5 за първи път не влизаме в тялото на цикъла. От инварианта имаме, че:
		\begin{itemize}
			\item $i=n+1$
			\item $curr\in\{1,\dots,n\}$
			\item $cnt>0$
			\item $t\in\{0,1\}$
			\item $\exists S\big(mcpset_{A[1..n]}(S)\land(cnt-t>0\Rightarrow curr\notin idx(S))\land(cnt-t+|idx(S)|=n)\big)$
		\end{itemize}
		Допускаме, че $cnt-t\ngtr0$, т.е. $cnt-t\le0$, но $cnt>0$ и $t\in\{0,1\}$ и значи $cnt=t=1$. Оттук $cnt-t=0$. Следователно $|idx(S)|=n$. Сега, тъй като $mcpset_{A[1..n]}(S)$, в частност $cpset_{A[1..n]}(S)$, то $idx(S)=\{1,\dots,n\}$. От друга страна имаме по условие, че $A[1..n]$ има мажоранта. Тогава от точка едно на $\propref{prop:majority-element}$ заключаваме, че $idx(S)\ne\{1,\dots,n\}$, което е абсурд. Следователно $cnt-t>0$. Тогава $curr\notin idx(S)$, но $curr\in\{1,\dots,n\}$, т.е. $curr\in\{1,\dots,n\}\setminus idx(S)$. Отново се възползваме от максималността на $S$ и от точка две на $\propref{prop:majority-element}$ заключаваме, че $A[curr]$ е мажорантата, което е точно връщаната стойност, директно след цикъла.
	\end{termination}
	
	\noindent
	Сега остана да съставим алгоритъм, който да установява дали $A[1..n]$ има мажоранта:
	\begin{pseudocode}
		\SetKwData{dA}{A}
		\SetKwData{dn}{n}
		\SetKwData{dm}{m}
		\SetKwData{dcnt}{cnt}
		\SetKwData{di}{i}
		
		$ContainsMajorityElement(\dA[1..\dn])://\,\dA\in\mathbb{Z}^n,\dn\in\mathbb{N}^+$
		\Mybegin
		{	
			$\dm\leftarrow MajorityElementExpanded(\dA[1..\dn])$\;
			$\dcnt\leftarrow0$\;
			\Myfor{$\di\leftarrow1$ $\KwTo$ $\dn$}
			{
				\If{$\dA[\di]=\dm$}{$\dcnt\leftarrow\dcnt+1$\;}
			}
			\KwRet{$\dcnt\ge\lfloor\frac{\dn}2\rfloor+1$\;}
		}
	\end{pseudocode}
	Докажете коректността за упражнение. Сложността по време и на трите алгоритъма е $\Theta(n)$.
\end{solution}\newpage

\subsection{Амортизирана сложност (по време)}

Няма да даваме формална дефиниция на понятието, тъй като не е по конспекта на курса. Въпреки това, всеки себеуважаващ се програмист, трябва да знае какво представлява амортизираната сложност, поне на интуитивно ниво. Амортизираната сложност (по време) се среща предимно при алгоритми от тип заявка. Цялата идея е да получим средната (в смисъл на математическо очакване) сложност по време на заявката. Нека да разгледаме пример.\newline\newline

\begin{examplecp}
	Разглеждаме $std::vector::push\_back$. Нека имаме вектор $v[1..4]=[1,2,3,4]$ и $v.capacity()=4$, където $vector::capacity$ ни връща заделената памет. Сега ще разгледаме какво се случва, когато добавяме елементи.
	\begin{itemize}
		\item $v.push\_back(5)$
		\begin{itemize}
			\item няма алокирана памет, така че алокираме 8 единици памет: $[]\to[\_,\_,\_,\_,\_,\_,\_,\_]$
			\item копираме елементите: $[\_,\_,\_,\_,\_,\_,\_,\_]\to[1,2,3,4,\_,\_,\_,\_]$
			\item добавяме новия елемент: $[1,2,3,4,\_,\_,\_,\_]\to[1,2,3,4,5,\_,\_,\_]$
			\item трием старата памет: $[1,2,3,4]\to[]$
		\end{itemize}
		Общо време: $8+4+1+4=17$ единици работа.
		
		\item $v.push\_back(6)$
		\begin{itemize}
			\item добавяме новия елемент: $[1,2,3,4,5,\_,\_,\_]\to[1,2,3,4,5,6,\_,\_]$
		\end{itemize}
		Общо време: $1$ единица работа.
		
		\item $v.push\_back(7)$
		\begin{itemize}
			\item добавяме новия елемент: $[1,2,3,4,5,6,\_,\_]\to[1,2,3,4,5,6,7,\_]$
		\end{itemize}
		Общо време: $1$ единица работа.
		
		\item $v.push\_back(8)$
		\begin{itemize}
			\item добавяме новия елемент: $[1,2,3,4,5,6,7,\_]\to[1,2,3,4,5,6,7,8]$
		\end{itemize}
		Общо време: $1$ единица работа.
	\end{itemize}
	Общо време за добавяне на $5,6,7$ и $8$: $17+1+1+1=20$ едциници работа.
	
	
	\begin{itemize}
		\item $v.push\_back(9)$
		\begin{itemize}
			\item няма алокирана памет, така че алокираме 16 единици памет: $[]\to[\_,\dots,\_]$ //16
			\item копираме елементите: $[\_,\dots,\_]\to[1,\dots,8,\_,\_,\_,\_,\_,\_,\_,\_]$
			\item добавяме новия елемент: $[1,\dots,8,\_,\_,\_,\_,\_,\_,\_,\_]\to[1,\dots,8,9,\_,\_,\_,\_,\_,\_,\_]$
			\item трием старата памет: $[1,2,3,4,5,6,7,8]\to[]$
		\end{itemize}
		Общо време: $16+8+1+8=33$ единици работа.
		
		\item $v.push\_back(10)$
		\begin{itemize}
			\item добавяме новия елемент: $[1,\dots,8,9,\_,\_,\_,\_,\_,\_,\_]\to[1,\dots,8,9,10,\_,\_,\_,\_,\_,\_]$
		\end{itemize}
		Общо време: $1$ единица работа.
		\begin{center}
			\dots
		\end{center}
		\item $v.push\_back(16)$
		\begin{itemize}
			\item добавяме новия елемент: $[1,\dots,15,\_]\to[1,\dots,15,16]$
		\end{itemize}
		Общо време: $1$ единица работа.
	\end{itemize}
	Общо време за добавяне на $9,10,\dots,16$: $33+\underbrace{1+1+\dots+1}_7=40$ едциници работа.\\
	Аналогично, общото време за добавяне на $17,\dots,32$ отнема $80$ единици работа и т.н. Може да забележим, че някои $push$-вания са бавни (линейни спрямо $n$), но те се срещат рядко. Повечето $push$-вания са константи. Сега взизмаме средното им време (по групички):
	\begin{itemize}
		\item $[5,8]$
		
		Има общо $4$ елемента и отнема общо $20$ единици работа. Значи средно на $push$-ване е $\frac{20}4=5$ единици работа.
		
		\item $[9,16]$
		
		Има общо $8$ елемента и отнема общо $40$ единици работа. Значи средно на $push$-ване е $\frac{40}8=5$ единици работа.
		
		\item $[17,32]$
		
		Има общо $16$ елемента и отнема общо $80$ единици работа. Значи средно на $push$-ване е $\frac{80}{16}=5$ единици работа.
		
		\begin{center}
			\dots
		\end{center}
	\end{itemize}
	
	\noindent
	Това осреднено време на заявка, наричаме амортизирана сложност (по време). Тоест амортизираната сложност на $std::vector::push\_back$ е $\Theta_{a}(1)$, където индексът $a$ идва от $amortized$ (локално означение). Разбира се, амортизираната сложност би била безсмислена, ако направим малък брой заявки, тъй като можем да попаднем на "бавна"\ заявка. Има смисъл да се разглежда \textbf{само} в контекста на много заявки. Забележете също така, че бавна заявка ни гарантира много на брой бързи заявки. Това ни гарантира, че не можем да попадаме много пъти (или всички пъти) в "лошия"\ случай. В рандомизираните алгоритми (каквото и да е това) нямаме такава гаранция!
\end{examplecp}