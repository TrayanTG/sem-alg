\section{Уводни алгоритми}

Ще започнем с това да изтъкнем разликата между изчислителна задача и алгоритъм. Грубо казано, изчислителна задача е нещо общо, а алгоритъмът е прозрачна кутия (т.е. знаем ѝ механизъмът на работа), която \emph{"решава"} дадена изчислителна задача. Може да имаме много алгоритми, решаващи една и съща изчислителна задача. За да правим разлика, когато говорим за изчислителна задача ще използваме термините \emph{екземпляр (instance)} и \emph{решение (solution)} вместо \emph{вход} и \emph{изход}. Вторите ще използваме, когато говорим за алгоритъм.\\

\begin{problem}
	Даден е масив $A[1..n]\in(\mathbb{N}_0)^n$ и много на брой заявки от вида:
	\cproblem{i,j\in\{1,\dots,n\}}{\sum\limits_{k=1}^jA[k]}
	Предложете двойка алгоритми (индекс и заявка), решаващи изчислителната задача. Индексирането бива извиквано точно веднъж. Заявки биват извиквани многократно. Заявка не може бъде извикана преди (единственото) извикване на индекса. Разгледайте следния пример, имащ две заявки:
	\cproblemquery{A[1..7]=[4,2,5,7,1,1,10]}{\query{i=2,j=5}{15\ //\text{since }2+5+7+1=15}\\\query{i=5,j=2}{0\ //\text{since the neutral element of }+\text{ is }0}}
\end{problem}

\begin{solution}
	Нека да разгледаме първо индекирането:
	\begin{pseudocode}
		\SetKwData{dA}{A}
		\SetKwData{dn}{n}
		\SetKwData{dSum}{Sum}
		\SetKwData{di}{i}
		
		$Index(\dA[1..\dn])://\,\dA\in(\mathbb{N}_0)^{\dn},\dn\in\mathbb{N}^+$
		\Mybegin
		{
			$\dSum[1..\dn]\leftarrow[0,\dots,0]$\;
			$\dSum[1]\leftarrow\dA[1]$\;
			\Myfor{$i\leftarrow1$ $\KwTo$ $\dn$}
			{
				$\dSum[\di]\leftarrow\dSum[\di-1]+\dA[\di]$\;
			}
			\KwRet{$\dSum[1..\dn]$}\;
		}
	\end{pseudocode}
	След като направихме индекс (връщаният масив $Sum[1..n]$), трябва да го използваме по подходящ начин в заявката:
	\begin{pseudocode}
		\SetKwData{dn}{n}
		\SetKwData{dSum}{Sum}
		\SetKwData{di}{i}
		\SetKwData{dj}{j}
		
		$Query(\dSum[1..\dn],\di,\dj)://\,\dSum\in(\mathbb{N}_0)^{\dn},\di,\dj\in\{1,\dots,\dn\},\dn\in\mathbb{N}^+$
		\Mybegin
		{
			$\dSum[1..\dn]\leftarrow[0,\dots,0]$\;
			
			\If{$\di<1$ \Or $\dj>\dn$}{\KwRet{$-1$};}
			\If{$\di>\dj$}{\KwRet{$0$};}
			\If{$\di=1$}{\KwRet{$\dSum[\dj]$};}
			
			\KwRet{$\dSum[\dj]-\dSum[\di-1]$}\;
		}
	\end{pseudocode}
	Алгоритмите са със сложност по време съответно $\langle\underbrace{\ \ \ \Theta(n)\ \ \ }_{Index(A[1..n])},\underbrace{\quad\quad\Theta(1)\quad\quad}_{Query(Sum[1..n],i,j)}\rangle$.
\end{solution}\vspace{0.5cm}

\begin{problem}\label{prob-4-1}
	Даден е непразен сортиран масив $A[1..n]\in(\{0,1,\dots,n\})^n$ съставен от уникални елементи. Да се намери $k\in\{0,1,\dots,n\}$ такова, че $k$ не е елемент на $A[1..n]$. Предложете колкото можете по-бърз алгоритъм, решаващ изчислителната задача. Разгледайте следния пример:
	\cproblem{A[1..7]=[0,1,2,4,5,6,7]}{3\ //\text{since }3\text{ is not an element of }A[1..7]}
\end{problem}
\begin{solution}
	За разлика от предходната задача, тук нямаме много заявки, а точно една. С други думи може да обединим индекса и заявката в едно общо решение:
	\begin{pseudocode}
		\SetKwData{dA}{A}
		\SetKwData{dn}{n}
		\SetKwData{dleft}{left}
		\SetKwData{dright}{right}
		\SetKwData{dmid}{mid}
		
		$Task2(\dA[1..\dn])://\,\dA\in(\mathbb{N}_0)^n,\dn\in\mathbb{N}^+$
		\Mybegin
		{
			$\dleft\leftarrow1$\;
			$\dright\leftarrow\dn$\;
			
			\While{$\dleft<\dright$}
			{
				$\dmid\leftarrow\lfloor\frac{\dleft+\dright}2\rfloor$\;
				\If{$\dA[\dmid]+1=\dmid$}{$\dleft\leftarrow\dmid+1$\;}
				\Else{$\dright\leftarrow\dmid$\;}
			}
			\If{$\dA[\dn]+1=\dn$}{\KwRet{$\dleft$\;}}
			\KwRet{$\dleft-1$\;}
		}
	\end{pseudocode}
	Алгоритъмът е със сложност по време $\Theta(\log n)$.
\end{solution}\vspace{0.5cm}

\begin{problem}
	Даден е непразен масив $A[1..n]\in\mathbb{Z}^n$ съставен от уникални елементи. Да се намери множеството от наредени тройки $\{\langle i,j,k\rangle\!\in\!\{1,\dots,n\}^3\,|\,i\!<\!j\!<\!k\land A[i]\!+\!A[j]\!+\!A[k]\!=\!0\}$. Предложете колкото можете по-бърз алгоритъм, решаващ изчислителната задача. Разгледайте следния пример:
	\cproblem{A[1..7]=[-10,1,2,3,7,8,9]}{\langle1,2,7\rangle,\langle1,3,6\rangle,\langle1,4,5\rangle}
\end{problem}\newpage
\begin{solution}
	Отново, аналогично на $\probref{prob-4-1}$ нямаме нужда от отделен алгоритъм за индексация и заявка, тъй като имаме само една заявка:
	\begin{pseudocode}
		\SetKwData{dA}{A}
		\SetKwData{dn}{n}
		\SetKwData{dAns}{Ans}
		\SetKwData{di}{i}
		\SetKwData{dj}{j}
		\SetKwData{dk}{k}
		\SetKwData{dorder}{order}
		
		$Task3(\dA[1..\dn])://\,\dA\in\mathbb{Z}^{\dn},\dn\in\mathbb{N}^+$
		\Mybegin
		{
			$sort(\dA[1..\dn],\dorder=ascending)$\;
			$\dAns\leftarrow List.Init()$\tcp*{лист от наредени тройки}

			\Myfor{$\di\leftarrow1$ $\KwTo$ $\dn-2$}
			{
				\If{$\dA[\di]\ge0$}{\Break\;}
				$\dj\leftarrow\di+1$\;
				$\dk\leftarrow\dn$\;
				\While{$\dj<\dk$}
				{
					\If{$\dA[\di]+\dA[\dj]+\dA[\dk]=0$}
					{
						$\dAns.PushBack(\langle\di,\dj,\dk\rangle)$\;
						$\dj\leftarrow\dj+1$\;
						$\dk\leftarrow\dk-1$\;
					}
					\ElseIf{$\dA[\di]+\dA[\dj]+\dA[\dk]<0$}{$\dj\leftarrow\dj+1$\;}
					\Else{$\dk\leftarrow\dk-1$\;}
				}
			}
			\KwRet{$\dAns$}\;
		}
	\end{pseudocode}
	Алгоритъмът е със сложност по време $\Theta(n^2)$.
\end{solution}\leavevmode\newline

\begin{problem}
	Даден е непразен масив $A[1..n]\in\mathbb{R}^n$. Да се сортира възходящо масивът, използвайки единствено функцията $reverse\!:\!\mathbb{R}^n\!\times\!\{1,\dots,n\}\!\times\!\{1,\dots,n\}\!\to\!\mathbb{R}^n$, дефинирана по следния начин:
	\begin{equation*}
		reverse(A[1..n],i,j)\leftrightharpoons\begin{cases}
			\big[A[1],\dots,A[i-1],\underbrace{A[j],A[j-1],\dots,A[i+1],A[i]}_{\text{в обратен ред}},A[j+1],\dots,A[n]\big],\, i<j\\
			A[1..n],\, \text{иначе}\\
		\end{cases}
	\end{equation*}
\end{problem}
\begin{solution}
	Идеята е да вземем най-голям елемент и да го сложим на първа позиция, веднага след което на последната свободна такава (с други думи - сортираме отзад напред). Нека да разиграем малък пример. Нека $A[1..3]=[1,3,2]$. Първо поставяме $3$ на пъра позиция чрез $reverse([1,3,2],1,2)$, т.е. получаваме $[3,1,2]$. След това веднага го слагаме най-отзад чрез $reverse([3,1,2],1,3)$, т.е. получаваме $[2,1,3]$. Така вече най-големият (най-голям в общия случай) елемент си е на позицията. Сега намираме втория най-голям елемент и го слагаме на първа позиция, след което го слагаме на предпоследна позиция чрез последнователно извикване на $reverse([2,1,3],1,1)\mapsto[2,1,3]$ и $reverse([2,1,3],1,2)\mapsto[1,2,3]$. Сега откриваме третият най-голям елемент и го слагаме на първа позиция след което на трета позиция отзад напред (в случая е безсмислено разбира, т.е. последователно извикваме $reverse([1,2,3],1,1)\mapsto[1,2,3]$ и $reverse([1,2,3],1,1)\mapsto[1,2,3]$. Така масивът е сортиран. Както сме свикнали от C/C++ ще си дефинираме и използваме спомагателни функции - за чистота на кода и за по-лесно доказателство на коректността. Нека първо си дефинираме функция, намираща индекса на най-левия максимален елемент от префикс на масива:
	\begin{pseudocode}
		\SetKwData{dA}{A}
		\SetKwData{dn}{n}
		\SetKwData{dmaxIdx}{maxIdx}
		\SetKwData{di}{i}
		\SetKwData{dk}{k}
		
		$findIdxOfMax(\dA[1..\dn],\dk)://\,\dA\in\mathbb{R}^{\dn},\dk\in\{1,\dots,\dn\},\dn\in\mathbb{N}^+$
		\Mybegin
		{
			$\dmaxIdx\leftarrow1$\;
			
			\Myfor{$\di\leftarrow2$ $\KwTo$ $\dk$}
			{
				\If{$\dA[\di]>\dA[\dmaxIdx]$}{$\dmaxIdx\leftarrow\di$\;}
			}
			\KwRet{$\dmaxIdx$\;}
		}
	\end{pseudocode}
	Използвайки тази функция, вече можем чисто да напишем на код идеята от по-горе:
	\begin{pseudocode}
		\SetKwData{dA}{A}
		\SetKwData{dn}{n}
		\SetKwData{dmaxIdx}{maxIdx}
		\SetKwData{dk}{k}
		
		$Task4(\dA[1..\dn])://\,\dA\in\mathbb{R}^{\dn},\dn\in\mathbb{N}^+$
		\Mybegin
		{	
			\Myfor{$\dk\leftarrow\dn$ $\KwDownTo$ $2$}
			{
				$\dmaxIdx\leftarrow findIdxOfMax(\dA[1..\dn],\dk)$\;
				$reverse(\dA[1..\dn],1,\dmaxIdx)$\;
				$reverse(\dA[1..\dn],1,\dk)$\;
			}
			\KwRet{$\dA[1..\dn]$\;}
		}
	\end{pseudocode}
	Алгоритъмът е със сложност по време $\Theta(n^2)$.
\end{solution}\leavevmode\newline

\begin{problem}
	Нека е дадена непразна булева матрица $A[1..m][1..n]\in(\{0,1\}^n)^m$, чиито колонки са сортирани низходящо, т.е. $\big(\forall r\in\{1,\dots,m-1\}\big)\big(\forall c\in\{1,\dots,n\}\big)\big(A[r+1][c]\le A[r][c]\big)$. Да се намери височината на най-високата колонка от единици, т.е. $\max\{\sum_{r=1}^m\!A[r][c]\,|\,c\in\{1,\dots,n\}\}$. Предложете колкото можете по-бърз алгоритъм, решаващ изчислителната задача. Разгледайте следния пример:
	\cproblem{A[1..4][1..5]=\begin{bmatrix}
							1 & 0 & 1 & 1 & 0\\
							0 & 0 & 1 & 1 & 0\\
							0 & 0 & 1 & 0 & 0\\
							0 & 0 & 0 & 0 & 0
							\end{bmatrix}}{3\ //\text{since that is the highest row of ones}}
\end{problem}
\begin{solution}
	Идеята е много проста. Откриваме височината на първата колонка и я запазваме в променлива $h$. След това проверяваме дали втроата колонка би имала повече единици. Ако има, то преглеждаме нейната височина и обновяваме $h$. Наблюдението, което използваме е, че тъй като търсим максимална по височина колонка, то ако втората колонка има по-малка височина от първата, то директно я прескачаме. Така се движим "стъпаловидно"$\ $от горе-ляво към долу-дясно. За нагледно с червен цвят е отбелязан пътят, който ще "измине"$\ $алгоритъмът на примерния вход:
	\begin{center}
		$\begin{bmatrix}
			\textcolor{red}1 & 0 & 1 & 1 & 0\\
			\textcolor{red}0 & \textcolor{red}0 & \textcolor{red}1 & 1 & 0\\
			0 & 0 & \textcolor{red}1 & 0 & 0\\
			0 & 0 & \textcolor{red}0 & \textcolor{red}0 & \textcolor{red}0
		\end{bmatrix}$
	\end{center}
	Ясно е, че клетките над червената начупена линия няма нужда да бъдат разглеждани. Вече, въоражени с идея, сме готови да реализираме алгоритъм, решаващ изчислителната задача:
	\begin{pseudocode}
		\SetKwData{dA}{A}
		\SetKwData{dn}{n}
		\SetKwData{dm}{m}
		\SetKwData{di}{i}
		\SetKwData{dj}{j}
		
		$Task5(\dA[1..\dm][1..\dn])://\,\dA\in(\{0,1\}^{\dn})^{\dm},\dn,\dm\in\mathbb{N}^+$
		\Mybegin
		{	
			$\di\leftarrow0$\;
			\Myfor{$\dj\leftarrow1$ $\KwTo$ $\dn$}
			{
				\While{$\di<\dm$ \And $\dA[\di+1][\dj]=1$}{$\di\leftarrow\di+1$\;}
			}
			\KwRet{$\di$\;}
		}
	\end{pseudocode}
	Алгоритъмът е със сложност по време $\Theta(n)+O(m)=O(n+m)$.
\end{solution}\vspace{0.35cm}

\begin{problem}
	Нека е дадена непразна матрица $A[1..m][1..n]\in((\mathbb{N}_0)^n)^m$ от уникални елементи, със свойството, че всеки елемент е по-голям от северните и западните си съседи. Казано формално $\forall r\forall c\big(A[i][j]<A[i+1][j]\land A[i][j]<A[i][j+1]\big)$, за $r$ и $c$ подходящи: да не излизаме извън дефиниционната област. Нека също така е дадено число $k\in\mathbb{N}_0$. Да се провери дали елемента $k$ се среща в матрицата $A[1..m][1..n]$. Предложете колкото можете по-бърз алгоритъм, решаващ изчислителната задача. Разгледайте следния пример:
	\cproblem{A[1..4][1..5]=\begin{bmatrix}
							 1 &  2 &  7 & 12 & 14\\
							 3 &  6 &  8 & 16 & 17\\
							 5 & 10 & 11 & 20 & 21\\
							13 & 15 & 19 & 22 & 25
	\end{bmatrix},k=15}{\text{TRUE}}
\end{problem}
\begin{solution}
	Идеята е да започнем или в най-североизточният или най-югозападният край и стъпаловидно да търсим елемента. Нека за определеност започваме в североизточният ъгъл. Движим се надолу, когато търсения елемент е по-голям от текущия и наляво, ако търсеният елемент е по-малък. Разбира се, ако текущия и търсения елемент съвпадат връщаме \text{TRUE}. Нека да оцветим в червено пътят, който изминава алгоритъмът за да намери 15 в примерната матрица:
	\begin{center}
		$\begin{bmatrix}
			1 &  2 &  7 & 12 & \textcolor{red}{14}\\
			3 &  6 &  \textcolor{red}8 & \textcolor{red}{16} & \textcolor{red}{17}\\
			5 & 10 & \textcolor{red}{11} & 20 & 21\\
			13 & \textcolor{red}{15} & \textcolor{red}{19} & 22 & 25
		 \end{bmatrix}$
	\end{center}
	Защо тази идея работи се убедете за упражнение! Може да го докажете дори формално. Вече сме готови да реализираме алгоритъм, решаващ изчислителната задача:
	\begin{pseudocode}
		\SetKwData{dA}{A}
		\SetKwData{dm}{m}
		\SetKwData{dn}{n}
		\SetKwData{di}{i}
		\SetKwData{dj}{j}
		\SetKwData{dk}{k}
		
		$Task6(\dA[1..\dm][1..\dn],\dk)://\,\dA\in((\mathbb{N}_0)^n)^m,\dn,\dm\in\mathbb{N}^+$
		\Mybegin
		{
			$\di\leftarrow1$\;
			$\dj\leftarrow\dn$\;
			
			\While{$\di\le\dm$ \And $\dj\ge1$}
			{
				\If{$\dA[\di][\dj]=\dk$}{\KwRet{\True\;}}
				\ElseIf{$\dA[\di][\dj]>\dk$}{$\dj\leftarrow\dj-1$\;}
				\Else{$\di\leftarrow\di+1$\;}
			}
			\KwRet{\False\;}
		}
	\end{pseudocode}
	Алгоритъмът е със сложност по време $O(n+m)$.
\end{solution}\newpage

\subsection{Добавяне на фиктивни променливи}

Доказателството за коректност на алгоритъм (зададен чрез псевдокод) не винаги е възможно. Оказва се, че не ни достигат променливите, дадени ни в псевдокода. В зависимост от това, къде се е провалило доказателството, добавяме нови, $\textbf{фиктивни}$ променливи. Те не оказват никакво влияние на алгоритъма. Служат единствено за опорни точки в доказателството за коректност. Независимо дали алгоритъмът е итеративен или рекурсивен, такъв проблем може да настъпи. Сега ще се съсредоточим върху итеративни алгоритми.\\
\emph{Спорно е дали мястото на подсекцията е тук или като подсекция на $\secref{sec:iter-alg-correctness}$.}\newline

\begin{problem}
	Даден е непразен масив $A[1..n]\in\mathbb{Z}^n$. Да се състави колкото се може по-бърз алгоритъм, намиращ мажорантата на $A[1..n]$, ако такава има. Съставете втори алгоритъм, който да установява дали мажоранта има.
\end{problem}
\begin{solution}
	Преди да изложим какъв да е псевдокод, ще докажем спомагателно твърдение. Въвеждаме следните означения:
	\begin{boxlocaldefinition*}{}{}
		\begin{itemize}
			\item $pair(p)\overset{\text{def}}\leftrightarrow\big(|p|=2\big)$
			\item $pset(S)\overset{\text{def}}\leftrightarrow\big(\forall p\in S\big)\big(pair(p)\big)$
			\item $cpair_{A[1..n]}(i,j)\!\overset{\text{def}}\leftrightarrow\!\big(i\!\in\!\{1,\dots,n\}\big)\!\land\!\big(j\!\in\!\{1,\dots,n\}\big)\!\land\!\big(A[i]\!\ne\! A[j]\big)$ //i.e. contrary pair
			\item $comp_{A[1..n]}(\langle i,j\rangle,\langle i',j'\rangle)\!\overset{\text{def}}\leftrightarrow\!\big(i\ne i'\big)\!\land\!\big(i\ne j'\big)\!\land\!\big(j\ne i'\big)\!\land\!\big(j\ne j'\big)$ //i.e. compatible
			\item $cpset_{A[1..n]}(S)\overset{\text{def}}\leftrightarrow pset(S)\land\big(\forall\langle i,j\rangle\in S\big)\big(cpair_{A[1..n]}(i,j)\big)\land$
			
			\hspace{3.2cm}$\big(\forall p_1\in S\big)\big(\forall p_2\in S\big)\big(comp_{A[1..n]}(p_1,p_2)\big)$
			\item $mcpset_{A[1..n]}(S)\overset{\text{def}}\leftrightarrow cpset_{A[1..n]}(S)\land\forall i\forall j\big(\langle i,j\rangle\notin S\Rightarrow\lnot cpset_{A[1..n]}(S\cup\{\langle i,j\rangle\})\big)$
			\item $idx(S)\leftrightharpoons Dom(S)\cup Rng(S)$ //also known as field
		\end{itemize}
	\end{boxlocaldefinition*}
	\begin{remark*}
		Казано неформално:
		\begin{itemize}
			\item ц-чифтове правим на индекси от $A[1..n]$, съответстващи на различни елементи
			\item ц-чифтове $\langle i,j\rangle$ и $\langle i',j'\rangle$ наричаме съвместими, ако $|\{i,i',j,j'\}|=4$
			\item множество от съвместими ц-чифтове наричаме максимално, ако не можем да добавим ц-чифт, така че да остане съвместимо
		\end{itemize}
	\end{remark*}%\leavevmode\newline
	\vspace{0.3cm}
	\begin{boxproposition}{}{}
		Нека $A[1..n]$ има мажоранта $m$. Тогава:
		\begin{itemize}
			\item $\forall S\big(mcpset_{A[1..n]}(S)\Rightarrow idx(S)\ne\{1,\dots,n\}\big)$
			\item $\forall S\Big(mcpset_{A[1..n]}(S)\Rightarrow\big(\forall i\!\in\!\{1,\dots,n\}\!\setminus\! idx(S)\big)\big(A[i]=m\big)\Big)$
		\end{itemize}
	\end{boxproposition}
	\begin{remark*}
		Казано неформално:
		\begin{itemize}
			\item за всяко максимално множество от съвместими ц-чифтове, има нечифтосан индекс
			\item за всяко максимално множество от съвместими ц-чифтове, всеки нечифтосан индекс отговаря на елемент, равен на мажорантата
		\end{itemize}
	\end{remark*}

	\begin{proof}
		Нека $A[1..n]$ е произволен масив, съдържащ мажоранта $m$. Нека $S$ е множество: $mcpset_{A[1..n]}(S)$ и нека ознчим броят на срещанията на мажорантата с $k$. Нека $i_1,\dots,i_k$ са (точно) индексите на срещанията на мажорантата в $A[1..n]$. Допускаме, че индексите на всички срещания на мажорантата са ц-чифтосани, т.е. има $k$ различни (от $i_1,\dots,i_k$ и помежду си) индекси $j_1,\dots,j_k$ такива, че $\{\langle i_1,j_1\rangle,\dots,\langle i_k,j_k\rangle\}\subseteq S$. Тоест $A[1..n]$ се състои от поне $2k$ елементи, или казано с други думи $n\ge2k$. От друга страна, знаем че $k\ge\lfloor\frac n2\rfloor+1>\frac n2$, откъдето $2k>n$, което е абсурд. Следователно има поне един не ц-чифтосан индекс $i$, отговарящ на мажорантата. Нека сега допуснем, че има не ц-чифтосан индекс $j$, отговарящ на елемент, различен от мажорантата. Тогава $cpset_{A[1..n]}(S\cup\{\langle i,j\rangle\})$, което противоречи с максималността на $S$. 
	\end{proof}

	\vspace{0.3cm}

	\begin{examplecp}
		Нека $A[1..10]=[1,2,3,1,1,1,4,1,1,5]$. Примери за максимални множества от съвместими ц-чифрове са $S_1=\{\langle1,2\rangle,\langle3,4\rangle,\langle5,7\rangle,\langle6,10\rangle\},S_2=\{\langle2,3\rangle,\langle7,10\rangle\}$. В първия случай множеството от не ц-нечифтосаните индекси е $\{8,9\}$, а във втория е $\{1,4,5,6,8,9\}$. И за двете множества $S_1$ и $S_2$ имаме, че:
		\begin{itemize}
			\item имат не ц-чифтосан индекс, т.е. $idx(S_1)=\{1,2,3,4,5,6,7,10\}\ne\{1,\dots,10\}$ и $idx(S_2)=\{2,3,7,10\}\ne\{1,\dots,10\}$;
			\item всички не ц-чифтосани индекси отговарят на елемент, равен на мажорантата $m=1$, т.е. за $S_1$ имаме $A[8]=A[9]=m$, а за $S_2$ имаме $A[1]=A[4]=A[5]=A[6]=A[8]=A[9]=m$.
		\end{itemize}
	\end{examplecp}\noindent\newline

	\noindent
	Първоначално ще изложим алгоритъм, намиращ мажорантата, при условие, че входният масив съдържа мажоранта:
	\begin{pseudocode}
		\SetKwData{dA}{A}
		\SetKwData{dn}{n}
		\SetKwData{dcurr}{curr}
		\SetKwData{dcnt}{cnt}
		\SetKwData{di}{i}
		
		$MajorityElement(\dA[1..\dn])://\,\dA\in\mathbb{Z}^n,\dn\in\mathbb{N}^+,A\text{ has majority element}$
		\Mybegin
		{	
			$\dcurr\leftarrow1$\;
			$\dcnt\leftarrow1$\;
			\Myfor{$\di\leftarrow2$ $\KwTo$ $\dn$}
			{
				\If{$\dA[\dcurr]=\dA[\di]$}{$\dcnt\leftarrow\dcnt+1$\;}
				\Else
				{
					$\dcnt\leftarrow\dcnt-1$\;
					\If{$\dcnt=0$}
					{
						$\dcurr\leftarrow\di$\;
						$\dcnt\leftarrow1$\;
					}
				}
			}
			\KwRet{$\dA[\dcurr]$\;}
		}
	\end{pseudocode}
	Сега ще се опитаме да докажем коректността на алгоритъма (няма да успеем).
	\begin{boxinvariant*}{}{}
		При всяко $k$-то достигане на ред $4$ имаме, че
		\begin{itemize}
			\item $i=k+1$
			\item $curr\in\{1,\dots,i-1\}$
			\item $cnt>0$
			\item $\exists S\big(mcpset_{A[1..i-1]}(S)\land(curr\notin idx(S))\land(cnt+|idx(S)|=n)\big)$
		\end{itemize}
	\end{boxinvariant*}
	\begin{remark*}
		Казано неформално: има максимално множество от съвместими ц-чифтове, където индексите са на $A[1..i-1]$, в което $curr$ не е ц-чифтосан. Нещо повече, броят на не ц-чифтованите индекси е $cnt$.
	\end{remark*}
	\begin{base}
		При $k=1$-во достигане на ред 4 имаме, че $i=2$, $curr\!=\!1\!\in\!\{1,\dots,2-1\}\!=\!\{1,\dots,i-1\}$. Разглеждаме подмасивът $A[1..1]$. Той е съставен от един елемент. Тогава $S=\emptyset$ ни е свидетел. Наистина, $1\notin\emptyset$ и има $cnt=1>0$ не ц-чифтосани индекси. 
	\end{base}
	\begin{maintenance}
		Нека е вярно за някое $k$-то непоследно достигане на ред 4 и нека $S_{old}$ ни е свидетел, т.е. $mcpset_{A[1..i_{old}-1]}(S_{old})\land(curr_{old}\notin idx(S_{old}))\land(cnt_{old}+|idx(S_{old})|=n)$. Сега се изпълнява ($k+1$)-то тялото на цикъла. Имаме $i_{new}=i_{old}+1=k+2$.
		\begin{mycase}
			\item $(A[curr_{old}]=A[i_{old}])$

			\vspace{-0.35cm}
			Тогава $curr_{new}=curr_{old}\in\{1,\dots,i_{old}-1\}\subset\{1,\dots,\overbrace{i_{new}-1}^{i_{old}}\}$, $cnt_{new}=cnt_{old}+1$. От ИП имаме, че $S_{old}$ е максимално множество от съвместими ц-чифтове, където индексите са на $A[1..i_{old}-1]$, $curr$ не е ц-чифтосан и броят на не ц-чифтосаните индекси е $cnt_{old}>0$. Лесно се проверява, че $S_{old}$ е максимално множество от съвместими ц-чифтове, където индексите са на $A[1..i_{new}-1]$, $curr$ не е ц-чифтосан и броят на ц-чифтосаните елементи е $cnt_{new}>1$, т.е. $mcpset_{A[1..i_{new}-1]}(S_{old})\land(curr_{new}\notin idx(S_{old}))\land(cnt_{new}+|idx(S_{old})|=n)$. С други думи $S_{old}$ е търсеният свидетел.
			
			\item $(A[curr_{old}]\ne A[i_{old}])$
			
			Тогава $cnt_{new}=cnt_{old}-1$. От ИП имаме, че $S_{old}$ е максимално множество от съвместими ц-чифтове, където индексите са на $A[1..i_{old}-1]$, $curr$ не е ц-чифтосан и броят на не ц-чифтосаните индекси е $cnt_{old}>0$. Тъй като $S_{old}$ е максимално, то всички не ц-чифтовани индекси отговарят на елементи с една и съща стойност $A[curr_{old}]\ne A[i_{old}]$.
			\begin{mycase}
				\item $(cnt_{old}>1)$
				
				\vspace{-0.35cm}
				Тогава $curr_{new}=curr_{old}\in\{1,\dots,i_{old}-1\}\subset\{1,\dots,\overbrace{i_{new}-1}^{i_{old}}\}$. Нека $j\ne curr_{old}$ е индекс, който не е ц-чифтован. Такъв има поради горното разъждение и защото $cnt_{old}>1$. Дефинираме $S_{new}\leftrightharpoons S_{old}\cup\{\langle j,i_{old}\rangle\}$. Лесно се проверява, че $S_{new}$ е максимално множество от съвместими ц-чифтове, където индексите са на $A[1..i_{new}-1]$, $curr_{new}$ не е ц-чифтосан и броят на ц-чифтосаните елементи е $cnt_{new}=cnt_{old}-1>1-1=0$, т.е. $S_{new}$ е търсеният свидетел.
				
				\item $(cnt_{old}=1)$
				
				Тогава $curr_{new}=i_{old}\in\{1,\dots,i_{old}\}=\{1,\dots,i_{new}-1\}$. Единствените два не ц-чифтовани индекса на $A[1..i_{old}]$ са $curr$ и $i_{old}$ и имаме, че $A[curr]\ne A[i_{old}]$. Дефинираме $S_{new}\leftrightharpoons S_{old}\cup\{\langle curr,i_{old}\rangle\}$. Лесно се проверява, че $S_{new}$ е максимално множество от съвместими ц-чифтове, където индексите са на $A[1..i_{new}-1]$, но $curr_{new}=i_{old}$ е ц-чифтосан, както и броят на ц-чифтосаните елементи е $cnt_{new}=cnt_{old}-1>0-1=-1$, т.е. $cnt_{new}\ge0$. \textbf{Не успяхме да намерим свидетел}. В общия случай, такъв може да има, но ние просто да не сме успели да го намерим (в конкретната задача може да се докаже, че наистина няма).
			\end{mycase}
		\end{mycase}
	Това всъщност беше очакван резултат, тъй като при вход $A[1..4]=[1,2,1,1]$, още след първата итерация на цикъла ц-чифтосваме индексите $1$ и $2$ и оставаме без не ц-чифтовани индекси, където индексите са на масива $A[1..2]$.
	\end{maintenance}
\end{solution}