%Асимптотични нотации

\section{Теория}

Започваме с припомняне на две дефинции от математическия анализ.

\begin{boxdefinition}{Асимптотично неотрицателна функция}{asym-non-negative}
	Ще казваме, че $f:\mathbb{R}_0^+\to\mathbb{R}$ е асимптотично неотрицателна, т.с.т.к. $\big(\exists n_0\in\mathbb{N}_0\big)\big(\forall n\ge n_0\big)\big(f(n)\ge0\big)$, където $\mathbb{R}_0^+$ са неотрицателнителните реални числа.
\end{boxdefinition}

\begin{boxdefinition}{Асимптотично положителна функция}{asym-positive}\label{asym-positive}
	Ще казваме, че $f:\mathbb{R}_0^+\rightarrow\mathbb{R}$ е асимптотично положителна, т.с.т.к. $\big(\exists n_0\in\mathbb{N}_0\big)\big(\forall n\ge n_0\big)\big(f(n)>0\big)$.
\end{boxdefinition}

\noindent
С $\mathscr{F}^+$ и $\mathscr{F}_0$ ще означаваме съответно множествата от всички асимптотично положителни и множествата от всички асимптотично неотрицателни функции. Функции в тези класове ще имат семантика на $\emph{брой стъпки до приключване изпълнението на дадена програма}$, което очевидно няма как да ни върне отрицателен резултат.
Също така ще се интересуваме да сравняваме (асимптотично) такива функции. За целта въвеждаме пет (несобствени) класа от фунции спрямо асимптотиката на дадена функция.

\begin{boxdefinition}{Основни класове от функции, спрямо асимптотиката на фукнция}{}\label{bdef-asymp-classes}
	Нека $g\in\mathscr{F}^+$. Дефинираме следните пет класа:
	\begin{itemize}
		\item $O(g)\leftrightharpoons\{f\in\mathscr{F}^+|\big(\exists c>0\big)\big(\exists n_0\in\mathbb{N}_0\big)\big(\forall n\ge n_0\big)\big(0\le f(n)\le cg(n)\big)\}$
		\item $o(g)\leftrightharpoons\{f\in\mathscr{F}^+|\big(\forall c>0\big)\big(\exists n_0\in\mathbb{N}_0\big)\big(\forall n\ge n_0\big)\big(0\le f(n)\le cg(n)\big)\}$
		\item $\Omega(g)\leftrightharpoons\{f\in\mathscr{F}^+|\big(\exists c>0\big)\big(\exists n_0\in\mathbb{N}_0\big)\big(\forall n\ge n_0\big)\big(0\le cg(n)\le f(n)\big)\}$
		\item $\omega(g)\leftrightharpoons\{f\in\mathscr{F}^+|\big(\forall c>0\big)\big(\exists n_0\in\mathbb{N}_0\big)\big(\forall n\ge n_0\big)\big(0\le cg(n)\le f(n)\big)\}$
		\item $\Theta(g)\leftrightharpoons\{f\in\mathscr{F}^+|\big(\exists c_1\!>\!0\big)\big(\exists c_2\!>\!0\big)\big(\exists n_0\!\in\!\mathbb{N}_0\big)\big(\forall n\!\ge\! n_0\big)\big(0\!\le\! c_1g(n)\!\le\! f(n)\!\le\! c_2g(n)\big)\}$
	\end{itemize}
	
\end{boxdefinition}

\begin{remark*}
	Поради исторически причини е прието да се пише $f=O(g)$ вместо $f\in O(g)$. Аналогично и за другите четири класа.
\end{remark*}\newpage

\begin{example}
	$O(n^2)=\{n^2,100n^2,\frac{n^2}5,n,{10}^6n,1,5n^2-1000n-1000,\dots\}$.

	\noindent
	Нека разгледаме примерни свидетели $c$ и $n_0$ за $f=O(n^2)$ съответно за $g(n)=n^2$ и $f(n)$:
	\begin{itemize}
		\item $f(n)=n^2$
		
		Директно се вижда, че $c=1$ и $n_0=0$ ни вършат работа:
		
		$\big(\forall n\ge\underbrace0_{n_0}\big)\big(0\le \underbrace{n^2}_{f(n)}\le\underbrace1_c*\underbrace{n^2}_{g(n)}\big)$
		\item $f(n)=100n^2$
		
		Отново лесно се преценя, че $c=100$ и $n_0=0$ ни вършат работа:
		
		$\big(\forall n\ge\underbrace0_{n_0}\big)\big(0\le \underbrace{100n^2}_{f(n)}\le\underbrace{100}_c*\underbrace{n^2}_{g(n)}\big)$
		
		\item $f(n)=\frac{n^2}5$
		
		Отново $c=1$ и $n_0=0$ ни вършат работа (разбира се и $c=\frac15$ $n_0=0$ също):
		
		$\big(\forall n\ge\underbrace0_{n_0}\big)\big(0\le \underbrace{\frac{n^2}5}_{f(n)}\le\underbrace1_c*\underbrace{n^2}_{g(n)}\big)$
		
		\item $f(n)=5n^2-1000n-1000$
		
		Макар и сташно на първи поглед, отново лесно се преценя, че $c={10}^7$ и $n_0=0$ ни вършат работа (разбира се това не е минимален избор на $c$ и $n_0$.. интересува ни само да намерим кои да е):
		
		$\big(\forall n\ge\underbrace0_{n_0}\big)\big(0\le \underbrace{5n^2-1000n-1000}_{f(n)}\le\underbrace{{10}^7}_c*\underbrace{n^2}_{g(n)}\big)$
	\end{itemize}
\end{example}

\noindent
Упражнете се с другите четири класа.

\begin{boxnotation}{Основни релации над асимптотично положителни функции}{}
	Въвеждаме следните пет релации за удобство:
	\begin{itemize}
		\item $f\preccurlyeq g\overset{\text{def}}\leftrightarrow f=O(g)$
		\item $f\prec g\overset{\text{def}}\leftrightarrow f=o(g)$
		\item $f\succcurlyeq g\overset{\text{def}}\leftrightarrow f=\Omega(g)$
		\item $f\succ g\overset{\text{def}}\leftrightarrow f=\omega(g)$
		\item $f\asymp g\overset{\text{def}}\leftrightarrow f=\Theta(g)$ (асимптотично равенство с точност до константа)
	\end{itemize}
\end{boxnotation}

\begin{boxnotation}{Асимптотично равенство}{}
	$f\sim g\overset{\text{def}}\leftrightarrow\lim\limits_{n\to\infty}\frac{f(n)}{g(n)}=1$
\end{boxnotation}

\begin{boxmainproperties}{Асимптотични релации}{}\label{mprop-1}
Нека $f,g\in\mathscr{F}^+$. Тогава:
	\begin{enumerate}
		\item $f\,\sigma\,g\land g\,\sigma\,h\rightarrow f\,\sigma\,h,\,\sigma\in\{\preccurlyeq,\prec,\succcurlyeq,\succ,\asymp\}$
		\item $f\,\sigma\,f,\,\sigma\in\{\preccurlyeq,\succcurlyeq,\asymp\}$
		\item $f\preccurlyeq g\land g\preccurlyeq f\rightarrow f\asymp g$
		\item $f\asymp g\leftrightarrow g\asymp f$
		\item $f\preccurlyeq g\leftrightarrow g\succcurlyeq f\,(\text{аналогично }f\prec g\leftrightarrow g\succ f)$
		\item $f+g\asymp max(f,g)$\\
			  $\textbf{Док: }\frac{f(n)+g(n)}2\le max(f(n),g(n))\le f(n)+g(n)$
		\item $\lim\limits_{n\to\infty}\frac{f(n)}{g(n)}=0\leftrightarrow f\prec g\,(f=o(g))$\\
			  $\textbf{Док:}$ От деф. на лимит имаме: $\big(\forall\varepsilon>0\big)\big(\exists n_0\in\mathbb{N}_0\big)\big(\forall n\ge n_0\big)\big(-\varepsilon<\frac{f(n)}{g(n)}<\varepsilon\big)$. Сега умножаваме двете страни по $g(n)$ и получаваме $-g(n)*\varepsilon<0<f(n)<g(n)*\varepsilon$, откъдето имаме $\big(\forall\varepsilon>0\big)\big(\exists n_0\in\mathbb{N}_0\big)\big(\forall n\ge n_0\big)\big(0\le f(n)\le\varepsilon g(n)\big)$, което е точно дефиницията на $\hyperref[bdef-asymp-classes]{o(g)}$. Обратната посока е аналогична.
		\item $\lim\limits_{n\to\infty}\frac{f(n)}{g(n)}=c>0\rightarrow f\asymp g\,(f=\Theta(g))$\\
			  $\textbf{Док:}$ От деф. на лимит имаме: $\big(\forall\varepsilon>0\big)\big(\exists n_0\in\mathbb{N}_0\big)\big(\forall n\ge n_0\big)\big(c-\varepsilon<\frac{f(n)}{g(n)}<c+\varepsilon\big)$. Сега умножаваме двете страни по $g(n)$ и получаваме $g(n)*(c-\varepsilon)<f(n)<g(n)*(c+\varepsilon)$. Тъй като $c>0$, то е ясно че има $\varepsilon>0:c-\varepsilon>0$. Тоест получихме $\big(\exists\varepsilon>0\big)\big(\exists n_0\in\mathbb{N}_0\big)\big(\forall n\ge n_0\big)\big(0<\underbrace{(c-\varepsilon)}_{c_1}g(n)<f(n)<\underbrace{(c+\varepsilon)}_{c_2}g(n)\big)$, което е
			  \vspace{-0.72cm}%-1.5\baselineskip}%
			  
			  точно дефиницията на $\hyperref[bdef-asymp-classes]{\Theta(g)}$.
			  \vspace{-0.15cm}
			  
			  Обратната посока $\textbf{не}$ е вярна.
		\item Нека $g$ не е ограничена отгоре и нека $a>1$. Тогава:
			  \begin{enumerate}
			  	\item $f\prec g\rightarrow a^{f(n)}\prec a^{g(n)}$ (нестрогия аналог $\textbf{не}$ е верен)
			  	\item $log_a\,f(n)\prec log_a\,g(n)\rightarrow f(n)\prec g(n)$ (нестрогия аналог $\textbf{не}$ е верен)
			  \end{enumerate}
		\item $\big(\forall a>1\big)\big(\forall t>0\big)\big(\forall\varepsilon>0\big)\big(log_a^t(n)\prec n^\varepsilon\big)$
	\end{enumerate}
\end{boxmainproperties}

\begin{boxfact}{Апроксимация на Стирлинг}{}
	$n!=\sqrt{2\pi n}\big(\frac{n}{e}\big)^n(1+\frac{1}{12n}+\frac{1}{288n^2}+\dots)$
\end{boxfact}

\begin{boxcorollary}{Асимптотична апроксимация на Стирлинг}{stirling}%\label{bcor-stirling}
	$n!\sim\sqrt{2\pi n}\big(\frac{n}{e}\big)^n$
\end{boxcorollary}

\begin{application*}
	\leavevmode
	\begin{itemize}
		\item $log(n!)\asymp nlog(n)$ (докажете за упражнение)
		\item $\binom{2n}{n}=\frac{(2n)!}{n!n!}\sim\frac{\sqrt{4\pi n}\big(\frac{2n}{e}\big)^{2n}}{2\pi n\big(\frac{n}{e}\big)^{2n}}=\frac{2^{2n}n^{2n}}{\sqrt{\pi n}\,n^{2n}}=\frac{2^{2n}}{\sqrt{\pi n}}=\frac{4^n}{\sqrt{\pi n}}$
	\end{itemize}
\end{application*}

\begin{boxfact}{Логаритми и техните свойства}{}\label{fact-log-props}
	\begin{enumerate}
		\setcounter{enumi}{-1}
		\item $\big(\forall a\in\mathbb{R}^+\backslash\{1\}\big)\big(\forall b\in\mathbb{R}^+\big)\big(a^x=b\leftrightarrow x=log_a(b)\big)$
		\item $a^{log_a(b)}=b$
		\item $log_{a^n}(b^m)=\frac mnlog_a(b)$
		\item
		\begin{enumerate}
			\item $log_a(x)+log_a(y)=log_a(xy)$
			\item $log_a(x)-log_a(y)=log_a(\frac{x}{y})$
		\end{enumerate}
		\item $log_a(x)=\frac{log_b(x)}{log_b(a)},\,b\in\mathbb{R}^+\backslash\{1\}$
		\begin{enumerate}
			\item $log_a(b)=\frac{1}{log_b(a)}$
			\item $log_b(a)*log_a(x)=log_b(x)$
		\end{enumerate}
		\item $a^{log_b(x)}=x^{log_b(a)}$
	\end{enumerate}
\end{boxfact}\leavevmode\newline


\section{Задачи}

\begin{problem}
	Нека $p(x)=a_0x^k+\dots+a_k$ е асимптотично положителен полином (т.е. $a_0>0$). Да се докаже, че $p(n)\asymp n^k$.
\end{problem}
\begin{solution}
	$\lim\limits_{n\to\infty}\frac{p(n)}{n^k}=\lim\limits_{n\to\infty}\frac{a_0x^k+\dots+a_k}{n^k}=\underbrace{\lim\limits_{n\to\infty}\frac{a_0n^k}{n^k}}_{a_0}+\underbrace{\lim\limits_{n\to\infty}\frac{a_1n^{k-1}}{n^k}}_0+\dots+\underbrace{\lim\limits_{n\to\infty}\frac{a_n}{n^k}}_0=a_0>0$.\\
	От основно свойство $\hyperref[mprop-1]{8}$ следва, че $p(n)\asymp n^k$.
\end{solution}\leavevmode\newline

\begin{problem}
	Нека $k\in\mathbb{N}^+$. Да се докаже, че $\binom nk\asymp n^k$.
\end{problem}
\begin{solution}
	$\lim\limits_{n\to\infty}\frac{\binom nk}{n^k}=\lim\limits_{n\to\infty}\frac{n(n-1)\dots(n-k+1)}{n^kk!}=\frac1{k!}>0$.
	От основно свойство $\hyperref[mprop-1]{8}$ следва, че $\binom nk\asymp n^k$.
\end{solution}\leavevmode\newline

\begin{problem}
	Да се докаже, че $(n+1)^n\asymp n^n$.
\end{problem}
\begin{solution}
	$\lim\limits_{n\to\infty}\frac{(n+1)^n}{n^n}=\lim\limits_{n\to\infty}\big(\frac{n+1}n\big)^n=e>0$.
	От основно свойство $\hyperref[mprop-1]{8}$ следва, че $(n+1)^n\asymp n^n$.
\end{solution}\leavevmode\newline

\begin{problem}
	Нека $f(n)=\begin{cases}
	1&,\lfloor n\rfloor\equiv0(2)\\
	n^2&,\lfloor n\rfloor\equiv1(2)
	\end{cases}$ и $g(n)=n$. Да се докаже, че те са асимптотично несравними.
\end{problem}
\begin{solution}
	За упражнение...
\end{solution}\leavevmode\newline

\begin{problem}
	Вярна ли е следната импликация: $f=O(g)\rightarrow(f=o(g)\lor f=\Theta(g))$?
\end{problem}
\begin{solution}
	Не е вярна!	Проверете за упражнение със следния контрапример:
	
	\noindent
	$f(n)=\begin{cases}
	n&,\lfloor n\rfloor\equiv0(2)\\
	1/n&,\lfloor n\rfloor\equiv1(2)
	\end{cases}$ и $g(n)=n$.
\end{solution}\leavevmode\newline

\begin{problem}
	Да се сортират по асимптотика следните функции:
	\vspace{0.25cm}
	\noindent
	\begin{center}
		$\begin{array}{llll}
			f_1(n)=n^3\qquad          & f_2(n)=\sqrt n\qquad    & f_3(n)=log(n)\qquad    & f_4(n)=log^2(n)      \vspace{0.2cm}\\
			f_5(n)=log^{(2)}(n)\qquad & f_6(n)=n!\qquad         & f_7(n)=a^n\qquad       & f_8(n)=a             \vspace{0.2cm}\\
			f_9(n)=n^n\qquad          & f_{10}(n)=n^{-2}\qquad  & f_{11}(n)=n^2\qquad    & f_{12}(n)=n^{log(n)}
		\end{array}$
	\end{center}
\end{problem}
\vspace{0.25cm}
\begin{solution}
	\leavevmode\newline
	$n^n\succ n!\succ a^n\succ n^{log(n)}\succ n^3\succ n^2\succ\sqrt n\succ log^2(n)\succ log(n)\succ log^{(2)}(n)\succ a\succ n^{-2}$\\
	Ще опишем подробно решенията:
	\begin{enumerate}[label=\textbf{\arabic*.}]
		\item $\bm{n^n\succ n!}$
		
		От анализа знаем, че $\lim\limits_{n\to\infty}\frac{n!}{n^n}=0$. Тогава от основно свойство $\hyperref[mprop-1]{7}$ следва, че $n!\prec n^n$.
		
		
		\vspace{0.2cm}
		\item $\bm{n!\succ a^n}$
		
		Ще приложим основно свойство $\hyperref[mprop-1]{9}\dots$ т.е. логаритмуваме двете страни и получаваме
		\begin{equation*}
			log(n!)\,\textbf?\,log(a^n)
		\end{equation*}
		\begin{equation*}
			nlog(n)\,\textbf?\,nlog(a)
		\end{equation*}
		Знаем, че $\lim\limits_{n\to\infty}\frac{nlog(a)}{nlog(n)}=0$. Тогава от основно свойство $\hyperref[mprop-1]{7}$ следва, че $nlog(a)\prec nlog(n)$. Оттук и от основно свойство $\hyperref[mprop-1]{9}$ следва, че $a^n\prec n!$.
		
		
		\vspace{0.2cm}
		\item $\bm{a^n\succ n^{log(n)}}$
		
		Ще приложим основно свойство $\hyperref[mprop-1]{9}\dots$ т.е. логаритмуваме двете страни и получаваме
		\begin{equation*}
			log(a^n)\,\textbf?\,log(n^{log(n)})
		\end{equation*}
		\begin{equation*}
			nlog(a)\,\textbf?\,log(n)log(n)
		\end{equation*}
		От основно свойство $\hyperref[mprop-1]{10}$ имаме $log(n)log(n)\prec nlog(a)$. Оттук и от основно свойство $\hyperref[mprop-1]{9}$ следва, че $n^{log(n)}\prec a^n$.
		
		
		\vspace{0.2cm}
		\item $\bm{n^{log(n)}\succ n^3}$
		
		Ще приложим основно свойство $\hyperref[mprop-1]{9}\dots$ т.е. логаритмуваме двете страни и получаваме
		\begin{equation*}
			log(n^{log(n)})\,\textbf?\,log(n^3)
		\end{equation*}
		\begin{equation*}
			log^2(n)\,\textbf?\,3log(n)
		\end{equation*}
		Знаем, че $\lim\limits_{n\to\infty}\frac{3log(n)}{log^2(n)}=0$. Тогава от основно свойство $\hyperref[mprop-1]{7}$ следва, че $3log(n)\prec log^2(n)$. Оттук и от основно свойство $\hyperref[mprop-1]{9}$ следва, че $n^3\prec n^{log(n)}$.
		
		
		\vspace{0.2cm}
		\item $\bm{n^3\succ n^2}$
		
		 Знаем, че $\lim\limits_{n\to\infty}\frac{n^2}{n^3}=0$. Тогава от основно свойство $\hyperref[mprop-1]{7}$ следва, че $n^2\prec n^3$.


		\vspace{0.2cm}
		\item $\bm{n^2\succ\sqrt n}$
		
		Знаем, че $\lim\limits_{n\to\infty}\frac{\sqrt n}{n^2}=0$. Тогава от основно свойство $\hyperref[mprop-1]{7}$ следва, че $\sqrt n\prec n^2$.
		
		
		\vspace{0.2cm}
		\item $\bm{\sqrt n\succ log^2(n)}$
		
		От основно свойство $\hyperref[mprop-1]{9}$ директно следва, че $log^2(n)\prec\sqrt n$.

		
		\vspace{0.2cm}
		\item $\bm{log^2(n)\succ log(n)}$

		Знаем, че $\lim\limits_{n\to\infty}\frac{log(n)}{log^2(n)}=0$. Тогава от основно свойство $\hyperref[mprop-1]{7}$ следва, че $log(n)\prec log^2(n)$.
		
		
		\vspace{0.2cm}
		\item $\bm{log(n)\succ log(log(n))}$
		
		Полагаме $m=log(n)$ и получаваме
		\begin{equation*}
			m\,\textbf?\,log(m)
		\end{equation*}
		От основно свойство $\hyperref[mprop-1]{10}$ следва, че $log(m)\prec m$. Сега като върнем полагането получаваме $log(log(n))\prec log(n)$.
		
		
		\vspace{0.2cm}
		\item $\bm{log(log(n))\succ a}$
		
		Знаем, че $\lim\limits_{n\to\infty}\frac{a}{log^{(2)}(n)}=0$. Тогава от основно свойство $\hyperref[mprop-1]{7}$ следва, че $a\prec log^{(2)}(n)$.
		

		\vspace{0.2cm}
		\item $\bm{a\succ n^{-2}}$

		Знаем, че $\lim\limits_{n\to\infty}\frac{n^{-2}}{a}=0$. Тогава от основно свойство $\hyperref[mprop-1]{7}$ следва, че $n^{-2}\prec a$.
	\end{enumerate}
\end{solution}\leavevmode\newline

\begin{problem}
	Да се сортират по асимптотика следните функции:
	\vspace{0.25cm}
	\noindent
	\begin{center}
		$\begin{array}{llll}
			f_1(n)=(\sqrt2)^{log(n)}\qquad  & f_2(n)=n^3\qquad             & f_3(n)=n!\qquad                         & f_4(n)=(log(n))!              \vspace{0.2cm}\\
			f_5(n)=e^{-2ln(n)}\qquad        & f_6(n)=log^2(n)\qquad        & f_7(n)=log(n!)\qquad                    & f_8(n)=2^{2^n}                \vspace{0.2cm}\\
			f_9(n)=n^{\frac1{log(n)}}\qquad & f_{10}(n)=log^{(2)}(n)\qquad & f_{11}(n)=\big(\frac{3}{2}\big)^n\qquad & f_{12}(n)=n2^n                \vspace{0.2cm}\\
			f_{13}(n)=4^{log(n)}\qquad      & f_{14}(n)=(n+1)!\qquad       & f_{15}(n)=\sqrt{log(n)}\qquad           & f_{16}(n)=2^{\sqrt{2log(n)}}  \vspace{0.2cm}\\
			f_{17}(n)=n^{log(log(n))}\qquad & f_{18}(n)=log(n)\qquad       & f_{19}(n)=2^{log(n)}\qquad              & f_{20}(n)=(log(n))^{log(n)}
		\end{array}$
	\end{center}
\end{problem}
\vspace{0.25cm}
\begin{solution}
	\leavevmode\newline
	$e^{-2ln(n)}\prec n^{\frac{1}{log(n)}}\prec log^{(2)}(n)\prec \sqrt{log(n)}\prec log(n)\prec log^2(n)\prec 2^{\sqrt{2log(n)}}\prec (\sqrt2)^{log(n)}\prec 2^{log(n)}\prec log(n!)\prec 4^{log(n)}\prec n^3\prec (log(n))!\prec (log(n))^{log(n)}\asymp n^{log(log(n))}\prec \big(\frac{3}{2}\big)^n\prec n2^n\prec n!\prec (n+1)!\prec 2^{2^n}$ Ще опишем подробно решенията:
	\begin{enumerate}[label=\textbf{\arabic*.}]
		
		\item $\bm{e^{-2ln(n)}\prec n^{\frac{1}{log(n)}}}$
		
		Ще преобразуваме функциите използвайки основните свойства на логаритмите:
		\begin{equation*}
			(e^{ln(n)})^{-2}\,\textbf?\,n^{log_n(2)}
		\end{equation*}
		\begin{equation*}
			(n^{ln(e)})^{-2}\,\textbf?\,2^{log_n(n)}
		\end{equation*}
		\begin{equation*}
			n^{-2}\,\textbf?\,2
		\end{equation*}
		Сега знаем, че $\lim\limits_{n\to\infty}\frac{n^{-2}}{2}=0$. Тогава от основно свойство $\hyperref[mprop-1]{7}$ следва, че $n^{-2}\prec2$.
		
		
		\vspace{0.2cm}
		\item $\bm{n^{\frac{1}{log(n)}}\prec log^{(2)}(n)}$
		
		Вече показахме, че $n^{\frac{1}{log(n)}}=2$. Сега знаем, че $\lim\limits_{n\to\infty}\frac2{log^{(2)}(n)}=0$. Тогава от основно свойство $\hyperref[mprop-1]{7}$ следва, че $2\prec log^{(2)}(n)$.
		
		
		\vspace{0.2cm}
		\item $\bm{log^{(2)}(n)\prec\sqrt{log(n)}}$
		
		Полагаме $m=log(n)$ и получаваме
		\begin{equation*}
			log(m)\,\textbf?\,\sqrt m
		\end{equation*}
		От основно свойство $\hyperref[mprop-1]{10}$ следва, че $log(m)\prec m$. Сега като върнем полагането получаваме $log^{(2)}(n)\prec\sqrt{log(n)}$.
		
		
		\vspace{0.2cm}
		\item $\bm{\sqrt{log(n)}\prec log(n)}$
		
		Директно прилагаме основно свойство $\hyperref[mprop-1]{7}$.
		
		
		\vspace{0.2cm}
		\item $\bm{log(n)\prec log^2(n)}$
		
		Директно прилагаме основно свойство $\hyperref[mprop-1]{7}$.
		
		
		\vspace{0.2cm}
		\item $\bm{(log(n))^2\prec2^{\sqrt{2log(n)}}}$
		
		Ще приложим основно свойство $\hyperref[mprop-1]{9}\dots$ т.е. логаритмуваме двете страни и получаваме
		\begin{equation*}
			log((log(n))^2)\,\textbf?\,log(2^{\sqrt{2log(n)}})
		\end{equation*}
		\begin{equation*}
			2log(log(n))\,\textbf?\,\sqrt{2log(n)}log(2)
		\end{equation*}
		Полагаме $m=log(n)$ и получаваме
		\begin{equation*}
			2log(m)\,\textbf?\,\sqrt{2m}
		\end{equation*}
		От основно свойство $\hyperref[mprop-1]{10}$ имаме $2log(m)\prec\sqrt{2m}$. Сега като върнем полагането имаме $2log(log(n))\prec\sqrt{2log(n)}$. Оттук и от основно свойство $\hyperref[mprop-1]{9}$ следва, че $(log(n))^2\prec2^{\sqrt{2log(n)}}$.
		
		
		\vspace{0.2cm}
		\item $\bm{2^{\sqrt{2log(n)}}\prec(\sqrt2)^{log(n)}}$
		
		Ще приложим основно свойство $\hyperref[mprop-1]{9}\dots$ т.е. логаритмуваме двете страни и получаваме
		\begin{equation*}
			log(2^{\sqrt{2log(n)}})\,\textbf?\,log(\sqrt n)
		\end{equation*}
		\begin{equation*}
			\sqrt{2log(n)}\,\textbf?\,\frac12log(n)
		\end{equation*}
		Знаем, че $\lim\limits_{n\to\infty}\frac{\sqrt{2log(n)}}{\frac12log(n)}=0$. Тогава от основно свойство $\hyperref[mprop-1]{7}$ следва, че $\sqrt{2log(n)}\prec \frac12log(n)$. Оттук и от основно свойство $\hyperref[mprop-1]{9}$ следва, че $2^{\sqrt{2log(n)}}\prec(\sqrt2)^{log(n)}$.
		
		
		\vspace{0.2cm}
		\item $\bm{(\sqrt2)^{log(n)}\prec2^{log(n)}}$
		
		Ще преобразуваме функциите използвайки основните свойства на логаритмите:
		\begin{equation*}
			n^{log(\sqrt2)}\,\textbf?\,n^{log(2)}
		\end{equation*}
		\begin{equation*}
			\sqrt n\,\textbf?\,n
		\end{equation*}
		Сега знаем, че $\lim\limits_{n\to\infty}\frac{\sqrt n}{n}=0$. Тогава от основно свойство $\hyperref[mprop-1]{7}$ следва, че $\sqrt n\prec n$.
		
		
		\vspace{0.2cm}
		\item $\bm{2^{log(n)}\prec log(n!)}$
		
		Вече показахме, че $2^{log(n)}=n$ и $log(n!)\asymp nlog(n)$. Сега знаем, че $\lim\limits_{n\to\infty}\frac n{nlog(n)}=0$. Тогава от основно свойство $\hyperref[mprop-1]{7}$ следва, че $n\prec nlog(n)$.
		
		
		\vspace{0.2cm}
		\item $\bm{log(n!)\prec 4^{log(n)}}$
		
		Ще преобразуваме $4^{log(n)}$ използвайки основните свойства на логаритмите:
		\begin{equation*}
			log(n!)\,\textbf?\,n^{log(4)}
		\end{equation*}
		\begin{equation*}
			nlog(n)\,\textbf?\,n^2
		\end{equation*}
		Тогава от основно свойство $\hyperref[mprop-1]{10}$ следва, че $nlog(n)\prec n^2$.
		
		
		\vspace{0.2cm}
		\item $\bm{4^{log(n)}\prec n^3}$
		
		Вече показахме, че $4^{log(n)}=n^2$. Сега знаем, че $\lim\limits_{n\to\infty}\frac{n^2}{n^3}=0$. Тогава от основно свойство $\hyperref[mprop-1]{7}$ следва, че $n^2\prec n^3$.
		
		
		\vspace{0.2cm}
		\item $\bm{n^3\prec(log(n))!}$
		
		Ще преобразуваме $(log(n))!$ използвайки $\hyperref[cor:stirling]{\text{апроксимацията на Стирлинг}}$:
		\begin{equation*}
			(log(n))!\sim\sqrt{2\pi log(n)}\bigg(\frac{log(n)}{e}\bigg)^{log(n)}
		\end{equation*}
		Ще приложим основно свойство $\hyperref[mprop-1]{9}\dots$ т.е. логаритмуваме двете страни и получаваме
		\begin{equation*}
			log(n^3)\,\textbf?\,log\Bigg(\sqrt{2\pi log(n)}\bigg(\frac{log(n)}{e}\bigg)^{log(n)}\Bigg)
		\end{equation*}
		\begin{equation*}
			3log(n)\,\textbf?\,\underbrace{log\big(\sqrt{2\pi log(n)}\big)}_{\asymp log(log(n))}+log(n)\,log(log(n))-log(n)log(e)
		\end{equation*}
		\begin{equation*}
			log(n)\,\textbf?\,log(n)\,log(log(n))
		\end{equation*}
		Знаем, че $\lim\limits_{n\to\infty}\frac{log(n)}{log(n)\,log(log(n))}=0$. Тогава от основно свойство $\hyperref[mprop-1]{7}$ следва, че $log(n)\prec log(n)\,log(log(n))$. Оттук и от основно свойство $\hyperref[mprop-1]{9}$ следва, че $n^3\prec(log(n))!$.
		
		
		\vspace{0.2cm}
		\item $\bm{(log(n))!\prec(log(n))^{log(n)}}$
		
		Полагаме $m=log(n)$ и получаваме
		\begin{equation*}
			m!\,\textbf?\,m^m
		\end{equation*}
		От анализа знаем, че $\lim\limits_{n\to\infty}\frac{m!}{m^m}=0$. Тогава от основно свойство $\hyperref[mprop-1]{7}$ следва, че $m!\prec m^m$.
		Сега като върнем полагането получаваме $(log(n))!\prec(log(n))^{log(n)}$.
		
		
		\vspace{0.2cm}
		\item $\bm{(log(n))^{log(n)}\asymp n^{log(log(n))}}$
		
		Директно прилагаме основно свойство $\hyperref[fact-log-props]{5}$ на логаритмите.
		

		\vspace{0.2cm}
		\item $\bm{n^{log(log(n))}\prec \big(\frac{3}{2}\big)^n}$

		Ще приложим основно свойство $\hyperref[mprop-1]{9}\dots$ т.е. логаритмуваме двете страни и получаваме
		\begin{equation*}
			log(n)\,log(log(n))\,\textbf?\,n
		\end{equation*}
		От основно свойство $\hyperref[mprop-1]{10}$ следва, че $log(n)\,log(log(n))\prec log^2(n)\prec n$. Оттук и от основно свойство $\hyperref[mprop-1]{9}$ следва, че $n^{log(log(n))}\prec \big(\frac{3}{2}\big)^n$.
		
		
		\vspace{0.2cm}
		\item $\bm{\big(\frac{3}{2}\big)^n\prec n2^n}$
		
		Директно прилагаме основно свойство $\hyperref[mprop-1]{7}$. //$\big(\frac{3}{2}\big)^n\prec 2^n\prec n2^n$
		
		
		\vspace{0.2cm}
		\item $\bm{n2^n\prec n!}$
		
		Ще приложим основно свойство $\hyperref[mprop-1]{9}\dots$ т.е. логаритмуваме двете страни и получаваме
		\begin{equation*}
			log(n2^n)\,\textbf?\,log(n!)
		\end{equation*}
		\begin{equation*}
			log(n)+nlog(2)\,\textbf?\,nlog(n)
		\end{equation*}
		Знаем, че $\lim\limits_{n\to\infty}\frac n{nlog(n)}=0$. Тогава от основно свойство $\hyperref[mprop-1]{7}$ следва, че $n\prec log(n)$. Оттук и от основно свойство $\hyperref[mprop-1]{9}$ следва, че $n2^n\prec n!$.
	
	
		\vspace{0.2cm}
		\item $\bm{n!\prec (n+1)!}$

		Директно прилагаме основно свойство $\hyperref[mprop-1]{7}$
		
		
		\vspace{0.2cm}
		\item $\bm{(n+1)!\prec2^{2^n}}$
		
		Ще приложим основно свойство $\hyperref[mprop-1]{9}\dots$ т.е. логаритмуваме двете страни и получаваме
		\begin{equation*}
			(n+1)log(n+1)\,\textbf?\,2^n
		\end{equation*}
		Логаритмуваме още веднъж
		\begin{equation*}
			log(n+1)\,log(log(n+1))\,\textbf?\,n
		\end{equation*}
		От основно свойство $\hyperref[mprop-1]{10}$ следва, че $log(n)\,log(log(n))\prec log^2(n)\prec n$. Оттук и от основно свойство $\hyperref[mprop-1]{9}$ следва, че $log(n)\,log(log(n))\prec2^n$. И отново от основно свойство $\hyperref[mprop-1]{9}$ следва, че $(n+1)!\prec2^{2^n}$.
	\end{enumerate}
\end{solution}