%add counter to subsubsection and it in the table of contents
\setcounter{tocdepth}{3}
\setcounter{secnumdepth}{3}

\newcommand{\doubleforall}{%
	\hspace{-0.65ex}%
	\mathrel{\ooalign{$\forall$\cr\hidewidth\raisebox{0.73ex}{\scalebox{0.6}[0.6]{\rule{1ex}{0.8pt}}}\hidewidth}}%
	\hspace{-0.65ex}%
}

\newcommand{\doubleexists}{%
	\hspace{-0.65ex}%
	\mathrel{\ooalign{$\exists$\cr\hidewidth\raisebox{0.05ex}{\scalebox{0.6}[0.8]{\hspace{0.8ex}\rule{0.8pt}{1.85ex}}}\hidewidth}}%
	\hspace{-0.65ex}%
}

\theoremstyle{definition}
\newtheorem{theorem}{Теорема}[chapter]
\newtheorem{definition}{Дефиниция}[chapter]
\newtheorem{problem}{{\bf Задача}}[chapter]
\newtheorem{proposition}{Твърдение}[chapter]
\newtheorem{corollary}[theorem]{Следствие}
\newtheorem{lemma}{Лема}[chapter]
\newtheorem{example}{Пример}[chapter]
\newtheorem*{examplecp}{Пример} %computational problem
\newtheorem*{solution}{Решение}
\newtheorem*{base}{База}
\newtheorem*{indhypothesis}{Индуктивна хипотеза}
\newtheorem*{indstep}{Индуктивна стъпка}
\newtheorem*{maintenance}{Поддръжка}
\newtheorem*{termination}{Терминация}


%\theoremstyle{remark}
\newtheorem{remark}{Забележка}[chapter]
\newtheorem*{remark*}{Забележка}%[chapter]
\newtheorem{application}{Приложение}[chapter]
\newtheorem*{application*}{Приложение}%[chapter]

\newcommand{\remref}[1]{\hyperref[#1]{\textbf{Забележка }}\textbf{\ref{#1}}}
\newcommand{\probref}[1]{\hyperref[#1]{\textbf{Задача }}\textbf{\ref{#1}}}
\newcommand{\thmref}[1]{\hyperref[#1]{\textbf{Теорема }}\textbf{\ref{#1}}}
\newcommand{\propref}[1]{\hyperref[#1]{\textbf{Твърдение }}\textbf{\ref{#1}}}
\newcommand{\exmref}[1]{\hyperref[#1]{\textbf{Пример }}\textbf{\ref{#1}}}
\newcommand{\aplref}[1]{\hyperref[#1]{\textbf{Приложение }}\textbf{\ref{#1}}}
\newcommand{\secref}[1]{\hyperref[#1]{\textbf{Секция }}\textbf{\ref{#1}}}
\newcommand{\zzrref}[1]{\hyperref[#1]{\textbf{ЗЗР }}\textbf{\ref{#1}}}



\newcommand{\cproblem}[2]{\begin{center}$\begin{cases}\textbf{Instance: }#1\\\textbf{Solution: }#2\end{cases}$\end{center}} %computable problem
\newcommand{\cproblemquery}[2]{\begin{center}$\begin{cases}\textbf{Given: }#1\\#2\end{cases}$\end{center}}
\newcommand{\query}[2]{\begin{cases}\textbf{Instance: }#1\\\textbf{Solution: }#2\end{cases}}
\newcommand{\dproblem}[2]{\begin{center}$\begin{cases}\textbf{Instance: }#1\\\textbf{Question: }#2\end{cases}$\end{center}} %decision problem

%remove "Chapter X" and "Section X".. decided to have them stay..
%\titleformat{\chapter}[display]{\normalfont\bfseries}{}{0pt}{\Large}
%\titleformat{\section}[display]{\normalfont\bfseries}{}{0pt}{\Large}

%useful colors
\definecolor{mycol1}{RGB}{203, 22, 243}
\definecolor{mybrown}{RGB}{102, 45, 0}
\definecolor{mygreen}{RGB}{0, 80, 40}
\definecolor{myblue}{RGB}{0,100,255}
\definecolor{mydarkblue}{RGB}{0,0,189}
\definecolor{myred}{RGB}{220,35,35}
\definecolor{mybordertheorem}{RGB}{80,45,22}
\definecolor{mybackgroundtheorem}{RGB}{244,227,215}
\definecolor{myborderfigure}{RGB}{68,85,00}
\definecolor{mybackgroundfigure}{RGB}{238,255,170}

%no idea what are those used for...
%\definecolor{myCgreen}{rgb}{0,0.6,0}
%\definecolor{myCgray}{rgb}{0.5,0.5,0.5}
%\definecolor{myCmauve}{rgb}{0.58,0,0.82}

%rename Глава to Семинар
%\makeatletter
%\renewcommand{\@chapapp}{Семинар}
%\makeatother

%something for the table of contents (to be as Minko's)
\makeatletter
\renewcommand*\l@section{\@dottedtocline{1}{1.42em}{2.7em}}
\makeatother

%something for the table of contents (to be as Minko's)
\makeatletter
\renewcommand*\l@subsection{\@dottedtocline{2}{4.086em}{3.2em}}
\makeatother

%style for the color box "Допълнение"
\tcbset{mytcbbox/.style={
		breakable, enhanced,
		enhanced jigsaw,
		colbacktitle=red!60!black,%black!1!white,
		fonttitle=\bfseries,coltitle=white,%mygreen,
		coltext=mygreen,colback=mygreen!2!white,colframe=red!60!black, arc is angular,
		attach boxed title to top center=
		{yshift=-0.25mm-\tcboxedtitleheight/2,yshifttext=2mm-\tcboxedtitleheight/2},
		boxed title style={enhanced,boxrule=0.5mm,
			frame code={ \path[tcb fill frame] ([xshift=-4mm]frame.west)
				-- (frame.north west) -- (frame.north east) -- ([xshift=4mm]frame.east)
				-- (frame.south east) -- (frame.south west) -- cycle; },
			interior code={ \path[tcb fill interior] ([xshift=-2mm]interior.west)
				-- (interior.north west) -- (interior.north east)
				-- ([xshift=2mm]interior.east) -- (interior.south east) -- (interior.south west)
				-- cycle;}} 
}}

%color box "Допълнение"
\newtcolorbox[auto counter,
number within=chapter,
list inside=myframe %used if in future needed to print table of contents for Допълнения
]{insertedframe}[2][]{%
	mytcbbox,
	list text = {#2},
	title=Допълнение \thetcbcounter: #2, #1,
}

%color box for everything else
%LABELS:
%the automatic label is <prefix> "th" (the last thing in curly brackets) then <separator> : (or change with label separator=<char>) and then <marker> (the random bonus curly brackets when making new boxtheorem).. <prefix><separator><marker>
%to use specific label when making new boxtheorem add in square brackets label=<labelname> in the begining like that:
%\begin{boxtheorem}[label=th-integral-criterion]{Интегрален критерий (частен случай)}{}
\newtcbtheorem[number within=chapter, list inside=mytheo]{boxtheorem}{Теорема}%
{reset, enhanced, every box,
	colback=black!2!white, colframe=black!85!white, coltext=black!85!white,
	fonttitle=\small\bfseries, drop fuzzy shadow}{th}
\newtcbtheorem[number within=chapter]{boxproposition}{Твърдение}%
{reset, enhanced, every box,
	colback=black!2!white, colframe=black!85!white, coltext=black!85!white,
	fonttitle=\small\bfseries, drop fuzzy shadow}{prop}
\newtcbtheorem[number within=chapter]{boxlemma}{Лема}%
{reset, enhanced, every box,
	colback=black!2!white, colframe=black!85!white, coltext=black!85!white,
	fonttitle=\small\bfseries, drop fuzzy shadow}{lem}
\newtcbtheorem[number within=chapter]{boxcorollary}{Следствие}%
{reset, enhanced, every box,
	colback=black!2!white, colframe=black!85!white, coltext=black!85!white,
	fonttitle=\small\bfseries, drop fuzzy shadow}{cor}
\newtcbtheorem[number within=chapter, list inside=mynotation]{boxnotation}{Нотация}%
{reset, enhanced, every box,
	colback=black!2!white, colframe=black!85!white, coltext=black!85!white,
	fonttitle=\small\bfseries, drop fuzzy shadow}{not}
\newtcbtheorem[number within=chapter]{boxdefinition}{Дефиниция}%
{reset, enhanced, every box,
	colback=black!2!white, colframe=black!85!white, coltext=black!85!white,
	fonttitle=\small\bfseries, drop fuzzy shadow}{def}
\newtcbtheorem[number within=chapter]{boxlocaldefinition}{Дефиниция (локална)}%
{reset, enhanced, every box,
	colback=black!2!white, colframe=black!85!white, coltext=black!85!white,
	fonttitle=\small\bfseries, drop fuzzy shadow}{ldef}
\newtcbtheorem[number within=chapter]{boxalternativedefinition}{Дефиниция (алтернативна)}%
{reset, enhanced, every box,
	colback=black!2!white, colframe=black!85!white, coltext=black!85!white,
	fonttitle=\small\bfseries, drop fuzzy shadow}{adef}
\newtcbtheorem[number within=chapter]{boxexample}{Пример}%
{reset, enhanced, every box,
	colback=black!2!white, colframe=black!85!white, coltext=black!85!white,
	fonttitle=\small\bfseries, drop fuzzy shadow}{exmpl}
\newtcbtheorem[number within=chapter]{boxconvention}{Конвенция}%
{reset, enhanced, every box,
	colback=black!2!white, colframe=black!85!white, coltext=black!85!white,
	fonttitle=\small\bfseries, drop fuzzy shadow}{conv}
\newtcbtheorem[number within=chapter]{boxproblem}{Задача}%
{reset, enhanced, every box,
	colback=black!2!white, colframe=black!85!white, coltext=black!85!white,
	fonttitle=\small\bfseries, drop fuzzy shadow}{probl}
\newtcbtheorem[number within=chapter, list inside=mycompproblem]{boxcompproblem}{Изч.\ Задача}%
{reset, enhanced, every box,
	colback=black!2!white, colframe=black!85!white, coltext=black!85!white,
	fonttitle=\small\bfseries, drop fuzzy shadow}{compprobl}
\newtcbtheorem[number within=chapter]{boxremark}{Забележка}%
{reset, enhanced, every box,
	colback=black!2!white, colframe=black!85!white, coltext=black!85!white,
	fonttitle=\small\bfseries, drop fuzzy shadow}{rem}
\newtcbtheorem[number within=chapter]{boxobservation}{Наблюдение}%
{reset, enhanced, every box,
	colback=black!2!white, colframe=black!85!white, coltext=black!85!white,
	fonttitle=\small\bfseries, drop fuzzy shadow}{obs}
\newtcbtheorem[number within=chapter]{boxclarification}{Разяснение}%
{reset, enhanced, every box,
	colback=black!2!white, colframe=black!85!white, coltext=black!85!white,
	fonttitle=\small\bfseries, drop fuzzy shadow}{clarification}
\newtcbtheorem[number within=chapter]{boxalgorithm}{Алгоритъм}%
{reset, enhanced, every box,
	colback=black!2!white, colframe=black!85!white, coltext=black!85!white,
	fonttitle=\small\bfseries, drop fuzzy shadow}{alg}
\newtcbtheorem[number within=chapter]{boxhypothesis}{Хипотеза}%
{reset, enhanced, every box,
	colback=black!2!white, colframe=black!85!white, coltext=black!85!white,
	fonttitle=\small\bfseries, drop fuzzy shadow}{hyp}
\newtcbtheorem[number within=chapter]{boxfact}{Факт}%
{reset, enhanced, every box,
	colback=black!2!white, colframe=black!85!white, coltext=black!85!white,
	fonttitle=\small\bfseries, drop fuzzy shadow}{fct}
\newtcbtheorem[]{boxzzr}{Задача за разпознаване}%
{reset, enhanced, every box,
	colback=black!2!white, colframe=darkblue!85!white, coltext=black!85!white,
	fonttitle=\small\bfseries, drop fuzzy shadow}{zzr}

%those two with different style..
\newtcbtheorem[number within=chapter]{boxinvariant}{Инвариант}%
{reset, enhanced, every box,
	colback=gold, colframe=black!85!white, coltext=black!85!white,
	coltitle=gold, width= 0.92 \linewidth, before=\par\smallskip\centering,after=\par,
	fonttitle=\small\bfseries, drop fuzzy shadow}{invar}
\newtcolorbox[number within=chapter, list inside=myreduction]{boxreduction}[2][]{detach title,
	before upper={Редукция~\thereductioncounter: \tcbtitle},
	colbacktitle=black,
	colback=white,coltitle=black,
	title={#2},#1}


%main properties.. removed the counter
\newtcbtheorem{boxmainproperties}{Основни свойства}%
{reset, enhanced, every box,
	colback=black!2!white, colframe=black!85!white, coltext=black!85!white,
	fonttitle=\small\bfseries, drop fuzzy shadow, theorem name}{mprops}



%environment for pseudocode (helper)
\newsavebox{\helpbox}
\newenvironment{colboxTre}[1]
{\newcommand\colboxcolor{#1}\begin{lrbox}{\helpbox}}
{\end{lrbox} \colorbox[HTML]{\colboxcolor}{\usebox{\helpbox}}}

%environment for pseudocode
\newenvironment{pseudocode}
{\setlength{\parskip}{-1cm}\SetNlSty{}{}{.}\begin{flushleft}\begin{colboxTre}{EFF0F1}\begin{algorithm}[H]}
{\end{algorithm}\end{colboxTre}\end{flushleft}\setlength{\parskip}{0cm}}

%pseudocode bonus things
\SetKwBlock{Mybegin}{}{}
\SetKwRepeat{Do}{do}{while}
\SetKwFor{Myfor}{for}{}{}
\SetKwFor{Myforstep}{for}{with step}{}
\SetKw{Withstep}{$\KwSty{with step}$}
\SetKw{KwDownTo}{downto}
\SetKw{True}{$\KwSty{TRUE}$}
\SetKw{False}{$\KwSty{FALSE}$}
\SetKw{Success}{$\KwSty{SUCCESS}$}
\SetKw{Fail}{$\KwSty{FAIL}$}
\SetKw{Or}{or}
\SetKw{And}{and}
\SetKw{Break}{break}
\SetKw{Choose}{choose}



%my cases - сл.2.1.1.3
\newlist{mycase}{enumerate}{4}
\setlist[mycase,1]{label=\textbf{сл.\arabic*}}
\setlist[mycase,2]{label=\textbf{сл.\arabic{mycasei}.\arabic*}}
\setlist[mycase,3]{label=\textbf{сл.\arabic{mycasei}.\arabic{mycaseii}.\arabic*}}
\setlist[mycase,4]{label=\textbf{сл.\arabic{mycasei}.\arabic{mycaseii}.\arabic{mycaseiii}.\arabic*}}
%next one is mycaseiv

%for flow chars (блок схеми)
\tikzstyle{io} = [ellipse, minimum width=3cm, minimum height=1cm, text centered, draw=black] %input-output box
\tikzstyle{st} = [rectangle, minimum width=3cm, minimum height=1cm, text centered, draw=black] %state box
\tikzstyle{if} = [diamond, minimum size=2cm, text centered, draw=black] %if box
\tikzstyle{arrow} = [thick, ->, >=stealth] %obviously :D
\tikzstyle{state-red} = [circle, minimum width=0.5cm, minimum height=0.5cm, text centered, draw=red, red] %for automa
\tikzstyle{state} = [circle, minimum width=0.5cm, minimum height=0.5cm, text centered, draw=black] %for automa
\tikzstyle{state-final} = [circle, minimum width=0.5cm, minimum height=0.5cm, text centered, draw=black, double] %for automa
